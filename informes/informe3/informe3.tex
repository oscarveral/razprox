% Archivo: informe1.tex

\documentclass[11pt,a4paper]{article}

% Idioma y tipografías
\usepackage[spanish, es-noquoting, es-lcroman, es-tabla]{babel}
\usepackage[T1]{fontenc}
\usepackage[utf8]{inputenc}
\usepackage{lmodern}

% Maquetación y tipografía fina
\usepackage[a4paper,margin=2.5cm]{geometry}
\usepackage{microtype}
\usepackage{setspace}
\onehalfspacing

% Utilidades
\usepackage{csquotes}
\usepackage{graphicx}
\usepackage{xcolor}
\usepackage{booktabs}
\usepackage{siunitx}
\usepackage{amsmath,amssymb}
\usepackage{enumitem}
\usepackage{float}
\setlist{nosep,leftmargin=*,labelsep=0.5em}

% Hipervínculos
\usepackage[hidelinks]{hyperref}
\hypersetup{
    pdftitle={Informe de Aprendizaje Difuso},
    pdfauthor={Juan Diego Gallego Nicolás, Óscar Vera López},
    pdfsubject={Aprendizaje Difuso},
    pdfkeywords={Zona de Vida de Holdridge, Lógica Difusa, Inferencia Difusa}
}

% Bibliografía
\usepackage[
    backend=biber,
    style=ieee,
    sorting=nyt,
    maxbibnames=99
]{biblatex}
\addbibresource{references.bib}

% Subfiguras
\usepackage{subcaption}

% Código (Python)
\usepackage{listings}
\lstset{
	language=Python,
	basicstyle=\ttfamily\small,
	keywordstyle=\color{blue},
	stringstyle=\color{red},
	commentstyle=\color{green!60!black}\itshape,
	numbers=left,
	numberstyle=\tiny,
	stepnumber=1,
	numbersep=5pt,
	showstringspaces=false,
	breaklines=true,
	frame=single,
	captionpos=b
}
% Código JSON
\lstdefinelanguage{JSON}{
    basicstyle=\normalfont\ttfamily,
    numbers=left,
    numberstyle=\tiny,
    stepnumber=1,
    numbersep=8pt,
    showstringspaces=false,
    breaklines=true,
    frame=single,
    literate=
     *{0}{{{\color{blue}0}}}{1}
      {1}{{{\color{blue}1}}}{1}
      {2}{{{\color{blue}2}}}{1}
      {3}{{{\color{blue}3}}}{1}
      {4}{{{\color{blue}4}}}{1}
      {5}{{{\color{blue}5}}}{1}
      {6}{{{\color{blue}6}}}{1}
      {7}{{{\color{blue}7}}}{1}
      {8}{{{\color{blue}8}}}{1}
      {9}{{{\color{blue}9}}}{1}
      {:}{{{\color{red}:}}}{1}
      {,}{{{\color{red},}}}{1}
      {\{}{{{\color{orange}\{}}}{1}
      {\}}{{{\color{orange}\}}}}{1}
      {[}{{{\color{orange}[}}}{1}
      {]}{{{\color{orange}]}}}{1}
      {"}{{{\color{green}"}}}{1}
}


% Comandos útiles

\begin{document}
\begin{titlepage}
    \centering
    % \includegraphics[height=2cm]{logo.png}\par\vspace{1cm} % Descomentar si hay logo
    {\Large Informe Técnico 3}\par\vspace{0.5cm}
    {\huge\bfseries Aprendizaje Difuso\par}\vspace{0.5cm}
    \begin{tabular}{@{}ll@{}}
        Autor 1: & Juan Diego Gallego Nicolás \\
        Contacto 1: & jdiego.gallego@um.es \\
        Autor 2: & Óscar Vera López \\
        Contacto 2: & oscar.veral@um.es \\
        Profesor: & Mercedes Valdés Vela \\
        Asignatura: & Conocimiento y Razonamiento Aproximado \\
    \end{tabular}
    % Include umu logo
    \vfill
    \includegraphics[height=8cm]{images/SelloUMU-negativo.png}\par\vspace{1cm}
    \vfill
    {\large \today}\par
\end{titlepage}

\pagenumbering{roman}
\clearpage

\tableofcontents
\clearpage
\pagenumbering{arabic}

\section{Introducción}
La clasificación de zonas bioclimáticas
proporciona una base para comprender la distribución de ecosistemas. En este contexto, el Sistema de Zonas de Vida de Holdridge
ofrece un marco robusto para categorizar regiones según sus características climáticas y geográficas.
Concretamente, este sistema utiliza tres magnitudes climáticas principales: la biotemperatura media anual (ABT),
la precipitación media anual (APP) y la relación de evapotranspiración potencial (PER).

\begin{figure}[H]
	\centering
	\includegraphics[width=0.65\textwidth]{images/triangle.png}
	\caption{Diagrama del sistema de Zonas de Vida de Holdridge \cite{WikiHoldridge}}
	\label{fig:holdridge-triangle}
\end{figure}

En la imagen anterior (\ref{fig:holdridge-triangle})
se muestra un diagrama representativo del sistema de Holdridge, donde se visualizan las diferentes zonas de vida en 
función de las tres magnitudes climáticas mencionadas.

\begin{figure}[H]
	\centering
	\includegraphics[width=0.6\textwidth]{images/pisos_altitudinales.png}
	\caption{ABT en función de la latitud y altitud \cite{INRENA1995}}
	\label{fig:abt-latitude}
\end{figure}

Holdridge también propone un método para aproximar la de ABT de un punto geográfico. Para ello,
establece una relación directa entre latitud y la ABT media anual a nivel del mar. A partir de esta relación,
se pueden calcular las desviaciones de la ABT en función de la altitud del punto geográfico con la regla 
empírica: la temperatura disminuye aproximadamente 6 °C por cada 1000 metros de ascenso en altitud (\ref{fig:abt-latitude}).

En este informe abordamos la tarea de aprendizaje difuso que permita la inferencia de la zona de vida de Holdridge correspondiente a un punto geográfico.
Obtendremos datos meteorológicos reales desde utilizando Meteostat \cite{Meteostat}, los cuales procesaremos e introduciremos en el motor de inferencia 
difusa desarrollado para el Informe 2 de la asignatura. A partir de estas inferencias,mediante el
software FIDv3.5 \cite{FID}, podremos construir un árbol de decisión difuso que permita la clasificación automática de zonas de vida de Holdridge.

\section{Preprocesamiento de datos meteorológicos}

En este informe trabajaremos con datos meteorológicos reales obtenidos de la plataforma Meteostat \cite{Meteostat} para la totalidad 
de la Unión Europea.

\subsection{Obtención de estaciones meteorológicas}

El primer paso realizado se ha codificado en Python en un script \texttt{shp2estaciones.py}, en el que dado un fichero shapefile
con los límites geográficos de la Unión Europea \cite{EUShp}, se obtiene el conjunto de estaciones meteorológicas disponibles en Meteostat
que se encuentran dentro de dichos límites. Para evitar sobrecarga de trabajo y ser prácticos, se ha dividido el área en una malla
de puntos equidistantes $0.0625$ grados en ambas direcciones, seleccionando la estación meteorológica más cercana a cada punto.

\begin{figure}[H]
    \centering
	\includegraphics[width=\textwidth]{images/estaciones_1.png}
	\caption{Subconjunto cargado desde Meteostat en la Unión Europea}
	\label{fig:total_estaciones}
\end{figure}

El mapa \ref{fig:total_estaciones} muestrea el conjunto de puntos seleccionado dentro de la Unión Europea. En azul el punto indica únicamanete 
una tierra firme, en rojo indica la presencia de una estación climática. No se muestra la totalidad de estaciones, sólo el subconjunto sobre
el que trabajamos; véase la \ref{text:partial} en la siguiente sección. Mediante este procedimiento se han 
obtenido más de 2500 estaciones meteorológicas, que hemos guardado junto a la longitud y latitud de cada punto utilizado en un fichero CSV.

\subsection{Obtención de datos meteorológicos históricos}

Una vez disponemos de las estaciones en CSV, mediante un segundo script en Python \texttt{estaciones2datosmet.py}, se obtienen,
para cada una de las muestras de estación-coordenadas, los datos meteorológicos históricos con frecuencia mensual desde el año 2015 
a 2020. Nótese que se descartan columnas innecesarias para este informe, quedándonos únicamente con las columnas de temperatura mínima,
temperatura máxima y precipitación.

\label{text:partial}

Se tiene que destacar una limitación práctica que hemos encontrado, que es la no disponibilidad de datos de forma total o parcial en 
ciertas estaciones. Se obtiene una división de 763 estaciones con datos completos,
300 estaciones con datos parciales interpolables, y 1900 estaciones sobre las que no se han podido obtener la totalidad de al menos
una de las columnas relevantes.

\begin{figure}[H]
    \centering
	\includegraphics[width=\textwidth]{images/estaciones_2.png}
	\caption{Completitud de datos meteorológicos históricos por estación}
	\label{fig:parcial_estaciones}
\end{figure}

La figura \ref{fig:parcial_estaciones} muestra la distribución geográfica de las estaciones con datos completos (azul) y datos parciales (rojo).
Como curiosidad, la API de Meteostat \cite{Meteostat} no nos ha permitido obtener datos de forma homogénea, predominando Alemania como fuente de
datos.

Una vez obtenidos estos datos, se construyen 2 ficheros CSV concatenando los datos originales de las estaciones y coordenadas 
con los datos meteorológicos históricos y relevantes  obtenidos, el primero contiene todas las estaciones con datos completos y 
el segundo contiene aquellas con datos parciales.

\subsection{Variables bioclimáticas}

A partir de estos ficheros, podemos calcular la biotemperatura media anual (ABT), la precipitación media anual (APP), y, finalmente, 
la relación de evapotranspiración potencial (PER) para cada uno de los puntos geográficos.

Respecto a la biotemperatura, calculamos la temperatura media mensual como la media aritmética de la temperatura máxima y mínima. Despúes,
calculamos la biotemperatura mensual aplicando la regla de Holdridge: si la temperatura media mensual es menor o igual a 0 °C, 
se asigna un valor de 0 °C, si está entre 0 °C y 30 °C, se mantiene el valor original, y si es mayor o igual a 30 °C, se asigna un valor de 30 °C.
Finalmente, la biotemperatura media anual (ABT) se obtiene como la media aritmética de las biotemperaturas anuales, tomadas como la media 
aritmetica de las biotemperaturas mensuales.

El calculo de la precipitación media anual (APP) es más sencillo, ya que se obtiene como la media aritmética de las precipitaciones anuales,
tomadas como la suma de las precipitaciones mensuales. Una vez se tiene la ABT y APP, se calcula la relación de evapotranspiración potencial (PER) 
utilizando la fórmula empírica propuesta por Holdridge:
$PER = \frac{ETP}{APP} = \frac{58.93 \times ABT}{APP}$, donde ETP es la evapotranspiración potencial anual.

Sobre los datos incompletos, hemos decidido interpolar los valores faltantes en las 300 estaciones para las que los datos faltantes son 
únicamente muestras sueltas a lo largo del tiempo de cualquiera de las columnas relevantes para este informe. Las otras 1900 estaciones con datos
incompletos han sido descartadas para el aprendizaje difuso, se utilizarán como muestras para evaluar como el sistema aprendido se comporta
ante datos reales con información incompleta.

Para el proceso de interpolación, se ha utilizado interpolación usando el método de polinomio de interpolación cúbica de hermite por tramos.
Para cada conjunto de muestras agrupadas anualmente, deducimos el valor que se debe escribir en los huecos de las columnas ABT y APP vacíos;
para después calcular el PER a partir de los valores interpolados. El proceso de interpolación no se aplica siempre, si un registro
anual tiene más de 3 meses con datos faltantes, no se realiza la interpolación y se considerea que la estación tiene datos incompletos.

De esta forma, obtendremos datos en los que, si se ha podido interpolar, se tendrán los valores de ABT, APP y PER completos para cada estación.
En otro caso, si sólo no se ha podido iterpolar ABT o APP, se descarta la estación como dato de entrenamiento. Se podrá intentar utilizar para
observar el comportamiento del sistema aprendido ante datos incompletos.

Después de fusionar los datos completos y los datos parciales interpolados, obtenemos un fichero CSV final más de 1000 muestras.

\subsection{Clasificación}

Una vez tenemos las variables bioclimáticas calculadas, podemos proceder a clasificar cada estación en su correspondiente zona de vida de Holdridge.
utilizando nuestro zonificador desarrollado en el Informe 2, quedándonos únicamente con la predicción de mayor grado de pertenencia.

\begin{figure}[H]
    \centering
	\includegraphics[width=\textwidth]{images/zonify_completas.png}
	\caption{Zonificación de estaciones con datos completos}
	\label{fig:zonify_completas}
\end{figure}

La Figura \ref{fig:zonify_completas} muestra la zonificación obtenida para las estaciones con datos completos. Para mejor visualización,
se han expandido los puntos en el mapa para cubrir un área mayor.

\begin{figure}[H]
    \centering
	\includegraphics[width=\textwidth]{images/zonify_parciales.png}
	\caption{Zonificación de estaciones con datos parciales}
	\label{fig:zonify_parciales}
\end{figure}

Véase como en la Figura \ref{fig:zonify_parciales}, los resultados utilizando datos interpolados son lo suficientemente coherentes con los obtenidos a partir de datos completos,
lo que nos permite utilizarlos para el aprendizaje difuso.

\begin{figure}[H]
    \centering
	\includegraphics[width=\textwidth]{images/zonify_fused.png}
	\caption{Zonificación de estaciones fusionadas}
	\label{fig:zonify_fused}
\end{figure}

\section{Aprendizaje difuso}

\subsection{Creación del conjunto de datos}

Para utilizar el software FID3.5 \cite{FID}, hemos tenido que transformar el fichero CSV final en un fichero de texto con el formato
esperado por el software. Para ello, hemos desarrollado un script en Python \texttt{zonify2fid2.4.py} que realiza dicha transformación,
obteniendo el fichero \texttt{.dat} que espera con muestras de entrenamiento.

Para aliviar el problema de sobre-representación e infra-representación de ciertas etiquetas en el conjunto de datos, hemos
ajustado el peso de las muestras en función de la frecuencia de aparición de cada etiqueta en el conjunto de datos. La fórmula utilizada
es la siguiente: $p = \sqrt{n}$, donde $p$ es el peso asignado a cada muestra y $n$ es el número de apariciones de la etiqueta 
correspondiente en el conjunto de datos. Se destaca que nuestro dataset de entrenamiento, al derivarse únicamente de datos
meteorológicos europeos, sólo contiene 22 de las 38 zonas de vida de Holdridge posibles en las hemos dividido el espacio de salida.
Recuerdese que nuestro zonificador es capaz de distingir entre todas las sub-zonas de vida resultantes de la 
subdivisión según la latitud. Por ejemplo, distinguir entre Bosque Húmedo Templado Frío y Bosque Húmedo Templado Cálido.

\subsection{Atributos y configuración}

Atendiendo a los primeros resultados preeliminares usando nuestro conocimiento experto, decidimos eliminar la variable de entrada PER, ya que observamos que no aportaba información adicional relevante
al estar correlacionada con las otras dos variables de entrada (ABT y APP). Incluso, llegaba a perjudicar a la generación
del árbol introduciendo perturbaciones. Sin PER, el árbol generado es más sencillo y con mejor capacidad de generalización.

Hemos explorado dos configuraciones diferentes para el particionado de las variables de entrada (ABT y APP):
\begin{itemize}
    \item Particionado experto: hemos definido automaticamente las particiones de los conjuntos difusos para cada una de las variables de entrada (ABT, APP)
    utilizando la configuración de nuestro motor de inferencia difusa del Informe 2.
    \item Particionado automático: hemos permitido que el software FID realice un particionado automático de las variables de entrada (ABT, APP),
    ajustando la tolerancia mínima de particionado a 0 y definiendo un rango de entre 7 y 13 conjuntos difusos para cada variable.
\end{itemize}

\subsection{Resultados obtenidos}

Inicialmente, utilizando el particionado experto, el árbol obtenido obtiene una precisión muy aceptable mayor al 
80\% validando los datos de entrenamiento. Observando la figura \ref{fig:confusion_experto} se aprecian leves
desviaciones fuera de la diagonal principal en la matriz de confusión, indicando que el sistema
comete algunos errores de clasificación. Por ejemplo, la Estepa Templada Fría, sólo se 
predicen correctamente la mitad de las veces, o, bosque boreal muy húmedo tiene un acierto del 0\%.
El mapa de clasificación resultante \ref{fig:mapa_experto} muestra bastante parecido a la zonificación
original obtenida con nuestro zonificador.

\begin{figure}[H]
    \centering
	\includegraphics[width=0.6\textwidth]{images/confusion_normal.png}
	\caption{Matriz de confusión con particionado experto}
	\label{fig:confusion_experto}
\end{figure}

\begin{figure}[H]
    \centering
	\includegraphics[width=0.6\textwidth]{images/mapa_normal.png}
	\caption{Mapa de clasificación con particionado experto}
	\label{fig:mapa_experto}
\end{figure}

Para comparar e intentar mejorar resultados, decidimos utilizar el particionado automático definido anteriormente.
De esta forma, el software FID es capaz de encontrar los particionados óptimos para cada variable de entrada, maximizando la precisión del árbol resultante.
Esta optimización automática ha permitido alcanzar una precisión del 79\% sobre el conjunto de datos de entrenamiento; menor a la obtenida con conocimiento
experto. Observando la matriz de confusión en la figura \ref{fig:confusion_automatico}, so observa una 
precisión por clase bastante buena, corrigiendo los errores graves de la clasificación anterior. Sin embargo, presenta
un ligerísimo incremento en los errores menores, lo que hace que obtenga una precisión global ligeramente menor. 
El mapa de clasificación resultante \ref{fig:mapa_automatico} muestra un parecido similar a la zonificación
original obtenida con nuestro zonificador y al mapa obtenido con particionado experto.

\begin{figure}[H]
    \centering
	\includegraphics[width=0.6\textwidth]{images/confusion_auto.png}
	\caption{Matriz de confusión con particionado automático}
	\label{fig:confusion_automatico}
\end{figure}

\begin{figure}[H]
    \centering
	\includegraphics[width=0.6\textwidth]{images/mapa_auto.png}
	\caption{Mapa de clasificación con particionado automático}
	\label{fig:mapa_automatico}
\end{figure}

Para mejorar aun más la precisión del sistema, se ha explorado la posibilidad de realizar un entrenamiento / construcción del árbol utilizando únicamente datos
sintéticos generados para cada zona de vida. En total, creamos sintéticamente 9 muestras representativas para cada zona de vida, y, manteniendo el particionado
manual inicial, construimos un árbol utilizando únicamente estos datos sintéticos. La generación de datos sintéticos
ha seguido el siguiente procedimiento (utilizando nuestro particionado experto inicial):
\begin{itemize}
    \item Para cada zona de vida, se obtienen las particiones difusas correspondientes a cada variable de entrada (ABT, APP).
    \item Para cada una de las particiones difusas, se obtiene 3 valores representativos.
    \item Se generan todas las combinaciones posibles de los valores representativos de cada variable de entrada.
    \item Cada combinación generada se etiqueta con la zona de vida correspondiente.
\end{itemize}
La única excepción en este procedimiento ha sido para las biotemperaturas extremas entre 0 y 1.5 ºC,
que corresponden siempre a la misma zona de vida (desierto) y por tanto serán fácilmente clasificables.

El árbol resultante mostró una precisión mayor al 82\% sobre los datos de 
entrenamiento, ahora tratados únicamente como datos de test. Véase en la figura \ref{fig:confusion_sinteticos} como la matriz de confusión muestra 
menos desviaciones notables fuera de la diagonal principal y como en el mapa de la figura \ref{fig:mapa_sinteticos} la clasificación es ya 
muy parecida la zonificación original obtenida con nuestro zonificador.

\begin{figure}[H]
    \centering
	\includegraphics[width=0.6\textwidth]{images/confusion_sint.png}
	\caption{Matriz de confusión con particionado experto y datos sintéticos}
	\label{fig:confusion_sinteticos}
\end{figure}

\begin{figure}[H]
    \centering
	\includegraphics[width=0.6\textwidth]{images/mapa_sint.png}
	\caption{Mapa de clasificación con particionado experto y datos sintéticos}
	\label{fig:mapa_sinteticos}
\end{figure}

Hasta ahora, hemos evaluado los resultados comparando las predicciones del árbol con la etiqueta de 
mayor pertenencia obtenida con nuestro zonificador. Para complementar los resultados, es interesante
observar cual es la tasa de acierto si tomamos en cuenta tambíen las segundas y terceras etiquetas
de mayor pertenencia obtenidas con nuestro zonificador; obteniendose los siguientes resultados:
\begin{itemize}
    \item Particionado experto. Coincidencias totales (1ª, 2ª o 3ª etiqueta): 96.89\%.
    \begin{itemize}
        \item Coincidencias con 1ª etiqueta: 80.32\%.
        \item Coincidencias con 2ª etiqueta: 8.76\%.
        \item Coincidencias con 3ª etiqueta: 7.82\%.
    \end{itemize}
    \item Particionado automático. Coincidencias totales (1ª, 2ª o 3ª etiqueta): 94.63\%.
    \begin{itemize}
        \item Coincidencias con 1ª etiqueta: 79.00\%.
        \item Coincidencias con 2ª etiqueta: 11.39\%.
        \item Coincidencias con 3ª etiqueta: 4.24\%.
    \end{itemize}
    \item Particionado experto con datos sintéticos. Coincidencias totales (1ª, 2ª o 3ª etiqueta): 99.11\%.
    \begin{itemize}
        \item Coincidencias con 1ª etiqueta: 82.67\%.
        \item Coincidencias con 2ª etiqueta: 8.66\%.
        \item Coincidencias con 3ª etiqueta: 7.82\%.
    \end{itemize}
\end{itemize}

Se puede observar como, al considerar las segundas y terceras etiquetas de mayor pertenencia,
la tasa de acierto aumenta significativamente en todas las configuraciones exploradas. En la 
configuración con datos sintéticos, llegamos a tener más de un 99\% de acierto.


A continuación se muestra el DICE-Score obtenido en cada una de las configuraciones exploradas:  
\begin{itemize}
    \item Particionado experto: 0.63.
    \item Particionado automático: 0.60.
    \item Particionado experto con datos sintéticos: 0.76. 
\end{itemize}
El DICE-Score mide la media armónica entre el accuracy y el recall, por lo que es una métrica
útil para evaluar el rendimiento del sistema en términos de precisión y exhaustividad.
Destáquese como, midiendo el DICE-Score, la configuración con datos sintéticos es 
la que mejor rendimiento obtiene con mucha diferencia.

\subsection{Reglas generadas}

Nuestos resultados iniciales con conocimiento experto, han dejado en evidencia como el atributo
PER no aportaba información adicional relevante para la clasificación de zonas de vida. El árbol
generado con particionado experto en la figura \ref{fig:reglas_experto} hace una primera
bifurcación basandose en el atributo ABT, y después utiliza APP para realizar las hojas finales.
Véase como para la menor biotemperatura posible, dado que tenemos pocos ejemplos en el conjunto
de datos, el árbol no llega a bifurcarse en APP, ya que con la biotemperatura ya es posible
inferir la zona de vida sin tener en cuenta el APP.

\begin{figure}[H]
    \centering
    \includegraphics[width=0.35\textwidth]{images/reglas_experto.png}
    \caption{Árbol de reglas generado con particionado experto}
    \label{fig:reglas_experto}
\end{figure}

Véase como en la figura \ref{fig:reglas_auto},
el árbol generado, dada la gran cantidad de hojas, resulta ser uno muy ancho. Se han generado
más subdivisiones en las variables de entrada, lo que permite una mayor especialización
de las reglas generadas. Sin embargo, el árbol resultante es más difícil de interpretar y
es posible que presente problemas de sobreajuste si no existen las hojas adecuadas
en ciertos caminos del árbol.

\begin{figure}[H]
    \centering
    \includegraphics[width=0.35\textwidth]{images/reglas_auto.png}
    \caption{Árbol de reglas generado con particionado automático}
    \label{fig:reglas_auto}
\end{figure}

Con los datos sintéticos y particionado experto, ahora el árbol en la figura 
\ref{fig:reglas_sint} contiene reglas para todas las zonas de vida posibles, lo que hace que
se generen todos los caminos posibles en el árbol, justificando la gran precisión obtenida.

\begin{figure}[H]
    \centering
    \includegraphics[width=0.35\textwidth]{images/reglas_sint.png}
    \caption{Árbol de reglas generado con particionado experto y datos sintéticos}
    \label{fig:reglas_sint}
\end{figure}

\subsection{Análisis de particiones}

La figura \ref{fig:particion_experta} muestra las particiones definidas manualmente para las variables de entrada,
junto a la distribución de dichas variables en el conjunto de datos de entrenamiento. Véase como la mayoría de las muestras
se concentran en unas pocas de las particiones, lo que afecta y dificulta la correcta clasificación de las zonas de vida.

\begin{figure}[H]
    \centering
	\includegraphics[width=0.6\textwidth]{images/particion_experta.png}
	\caption{Particion experta de variables de entrada}
	\label{fig:particion_experta}
\end{figure}

Es interesante descatar como, en la figura \ref{fig:particion_auto}, el particionado automático de conjuntos difusos, crea particiones más finas allí donde hay mayor densidad de muestras,
y particiones más amplias en zonas con menor densidad de muestras. Esto permitiría al sistema, en teoría, adaptarse mejor a la distribución real de los datos.

\begin{figure}[H]
    \centering
	\includegraphics[width=0.6\textwidth]{images/particion_auto.png}
	\caption{Particion automática de variables de entrada}
	\label{fig:particion_auto}
\end{figure}

Finalmente, véase como, cuando se utilizan datos sintéticos para el entrenamiento, las muestras
están completamente adaptadas a las particiones expertas; lo que permite un aprendizaje más diverso y,
por lo tanto, una mejor generalización del sistema aprendido.

\begin{figure}[H]
    \centering
	\includegraphics[width=0.6\textwidth]{images/particion_sint.png}
	\caption{Particion experta con datos sintéticos}
	\label{fig:particion_sint}
\end{figure}


\clearpage
\section*{Uso de IA}
Durante la elaboración de este informe, se ha utilizado ChatGPT-4 para asistir en la redacción a LaTeX de este documento
mediante la herramienta de OpenAI integrada en Visual Studio Code. También se ha mantenido una conversación 
con Gemini-2.5 de Google para obtener orientación y opiniones sobre la estructura y contenido del informe.
Junto a este documento se entrega un archivo HTML con el historial de la conversación mantenida con la misma.

\clearpage
% Bibliografía
\printbibliography

\end{document}