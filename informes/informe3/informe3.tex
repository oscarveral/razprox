% Archivo: informe1.tex

\documentclass[11pt,a4paper]{article}

% Idioma y tipografías
\usepackage[spanish, es-noquoting, es-lcroman, es-tabla]{babel}
\usepackage[T1]{fontenc}
\usepackage[utf8]{inputenc}
\usepackage{lmodern}

% Maquetación y tipografía fina
\usepackage[a4paper,margin=2.5cm]{geometry}
\usepackage{microtype}
\usepackage{setspace}
\onehalfspacing

% Utilidades
\usepackage{csquotes}
\usepackage{graphicx}
\usepackage{xcolor}
\usepackage{booktabs}
\usepackage{siunitx}
\usepackage{amsmath,amssymb}
\usepackage{enumitem}
\usepackage{float}
\setlist{nosep,leftmargin=*,labelsep=0.5em}

% Hipervínculos
\usepackage[hidelinks]{hyperref}
\hypersetup{
    pdftitle={Informe de Aprendizaje Difuso},
    pdfauthor={Juan Diego Gallego Nicolás, Óscar Vera López},
    pdfsubject={Aprendizaje Difuso},
    pdfkeywords={Zona de Vida de Holdridge, Lógica Difusa, Inferencia Difusa}
}

% Bibliografía
\usepackage[
    backend=biber,
    style=ieee,
    sorting=nyt,
    maxbibnames=99
]{biblatex}
\addbibresource{references.bib}

% Subfiguras
\usepackage{subcaption}

% Código (Python)
\usepackage{listings}
\lstset{
	language=Python,
	basicstyle=\ttfamily\small,
	keywordstyle=\color{blue},
	stringstyle=\color{red},
	commentstyle=\color{green!60!black}\itshape,
	numbers=left,
	numberstyle=\tiny,
	stepnumber=1,
	numbersep=5pt,
	showstringspaces=false,
	breaklines=true,
	frame=single,
	captionpos=b
}
% Código JSON
\lstdefinelanguage{JSON}{
    basicstyle=\normalfont\ttfamily,
    numbers=left,
    numberstyle=\tiny,
    stepnumber=1,
    numbersep=8pt,
    showstringspaces=false,
    breaklines=true,
    frame=single,
    literate=
     *{0}{{{\color{blue}0}}}{1}
      {1}{{{\color{blue}1}}}{1}
      {2}{{{\color{blue}2}}}{1}
      {3}{{{\color{blue}3}}}{1}
      {4}{{{\color{blue}4}}}{1}
      {5}{{{\color{blue}5}}}{1}
      {6}{{{\color{blue}6}}}{1}
      {7}{{{\color{blue}7}}}{1}
      {8}{{{\color{blue}8}}}{1}
      {9}{{{\color{blue}9}}}{1}
      {:}{{{\color{red}:}}}{1}
      {,}{{{\color{red},}}}{1}
      {\{}{{{\color{orange}\{}}}{1}
      {\}}{{{\color{orange}\}}}}{1}
      {[}{{{\color{orange}[}}}{1}
      {]}{{{\color{orange}]}}}{1}
      {"}{{{\color{green}"}}}{1}
}


% Comandos útiles

\begin{document}
\begin{titlepage}
    \centering
    % \includegraphics[height=2cm]{logo.png}\par\vspace{1cm} % Descomentar si hay logo
    {\Large Informe Técnico 3}\par\vspace{0.5cm}
    {\huge\bfseries Aprendizaje Difuso\par}\vspace{0.5cm}
    \begin{tabular}{@{}ll@{}}
        Autor 1: & Juan Diego Gallego Nicolás \\
        Contacto 1: & jdiego.gallego@um.es \\
        Autor 2: & Óscar Vera López \\
        Contacto 2: & oscar.veral@um.es \\
        Profesor: & Mercedes Valdés Vela \\
        Asignatura: & Conocimiento y Razonamiento Aproximado \\
    \end{tabular}
    % Include umu logo
    \vfill
    \includegraphics[height=8cm]{images/SelloUMU-negativo.png}\par\vspace{1cm}
    \vfill
    {\large \today}\par
\end{titlepage}

\pagenumbering{roman}
\clearpage

\tableofcontents
\clearpage
\pagenumbering{arabic}

\section{Introducción}
La clasificación de zonas bioclimáticas
proporciona una base para comprender la distribución de ecosistemas. En este contexto, el Sistema de Zonas de Vida de Holdridge
ofrece un marco robusto para categorizar regiones según sus características climáticas y geográficas.
Concretamente, este sistema utiliza tres magnitudes climáticas principales: la biotemperatura media anual (ABT),
la precipitación media anual (APP) y la relación de evapotranspiración potencial (PER).

\begin{figure}[H]
	\centering
	\includegraphics[width=0.65\textwidth]{images/triangle.png}
	\caption{Diagrama del sistema de Zonas de Vida de Holdridge \cite{WikiHoldridge}}
	\label{fig:holdridge-triangle}
\end{figure}

En la imagen anterior (\ref{fig:holdridge-triangle})
se muestra un diagrama representativo del sistema de Holdridge, donde se visualizan las diferentes zonas de vida en 
función de las tres magnitudes climáticas mencionadas.

\begin{figure}[H]
	\centering
	\includegraphics[width=0.6\textwidth]{images/pisos_altitudinales.png}
	\caption{ABT en función de la latitud y altitud \cite{INRENA1995}}
	\label{fig:abt-latitude}
\end{figure}

Holdridge también propone un método para aproximar la de ABT de un punto geográfico. Para ello,
establece una relación directa entre latitud y la ABT media anual a nivel del mar. A partir de esta relación,
se pueden calcular las desviaciones de la ABT en función de la altitud del punto geográfico con la regla 
empírica: la temperatura disminuye aproximadamente 6 °C por cada 1000 metros de ascenso en altitud (\ref{fig:abt-latitude}).

En este informe abordamos la tarea de aprendizaje difuso que permita la inferencia de la zona de vida de Holdridge correspondiente a un punto geográfico.
Obtendremos datos meteorológicos reales desde utilizando Meteostat \cite{Meteostat}, los cuales procesaremos e introduciremos en el motor de inferencia 
difusa desarrollado para el Informe 2 de la asignatura. A partir de estas inferencias, construiremos un set de datos etiquetados que podremos proporcionar al
software FIDv3.5 \cite{FID} para construir un árbol de decisión difuso que permita la clasificación automática de zonas de vida de Holdridge.

\section{Preprocesamiento de datos meteorológicos}

En este informe trabajaremos con datos meteorológicos reales obtenidos de la plataforma Meteostat \cite{Meteostat} para la totalidad 
de la Unión Europea.

\subsection{Obtención de estaciones meteorológicas}

El primer paso realizado se ha codificado en Python en un script \texttt{shp2estaciones.py}, en el que dado un fichero shapefile
con los límites geográficos de la Unión Europea \cite{EUShp}, se obtiene el conjunto de estaciones meteorológicas disponibles en Meteostat
que se encuentran dentro de dichos límites. Para evitar sobrecarga de trabajo y ser prácticos, se ha dividido el área en una malla
de puntos equidistantes $0.0625$ grados en ambas direcciones, seleccionando la estación meteorológica más cercana a cada punto.

\begin{figure}[H]
    \centering
	\includegraphics[width=\textwidth]{images/estaciones_1.png}
	\caption{Subconjunto cargado desde Meteostat en la Unión Europea}
	\label{fig:total_estaciones}
\end{figure}

El mapa \ref{fig:total_estaciones} muestrea el conjunto de puntos seleccionado dentro de la Unión Europea. En azul el punto indica únicamanete 
una tierra firme, en rojo indica la presencia de una estación climática. No se muestra la totalidad de estaciones, sólo el subconjunto sobre
el que trabajamos; véase la \ref{text:partial} en la siguiente sección. Mediante este procedimiento se han 
obtenido más de 2500 estaciones meteorológicas, que hemos guardado junto a la longitud y latitud de cada punto utilizado en un fichero CSV.

\subsection{Obtención de datos meteorológicos históricos}

Una vez disponemos de las estaciones en CSV, mediante un segundo script en Python \texttt{estaciones2datosmet.py}, se obtienen,
para cada una de las muestras de estación-coordenadas, los datos meteorológicos históricos con frecuencia mensual desde el año 2015 
a 2020. Nótese que se descartan columnas innecesarias para este informe, quedándonos únicamente con las columnas de temperatura mínima,
temperatura máxima y precipitación.

\label{text:partial}

Se tiene que destacar una limitación práctica que hemos encontrado, que es la no disponibilidad de datos de forma total o parcial en 
ciertas estaciones. Se obtiene una división de 763 estaciones con datos completos,
300 estaciones con datos parciales interpolables, y 1900 estaciones sobre las que no se han podido obtener la totalidad de al menos
una de las columnas relevantes.

\begin{figure}[H]
    \centering
	\includegraphics[width=\textwidth]{images/estaciones_2.png}
	\caption{Completitud de datos meteorológicos históricos por estación}
	\label{fig:parcial_estaciones}
\end{figure}

La figura \ref{fig:parcial_estaciones} muestra la distribución geográfica de las estaciones con datos completos (azul) y datos parciales (rojo).
Como curiosidad, la API de Meteostat \cite{Meteostat} no nos ha permitido obtener datos de forma homogénea, predominando Alemania como fuente de
datos.

Una vez obtenidos estos datos, se construyen 2 ficheros CSV concatenando los datos originales de las estaciones y coordenadas 
con los datos meteorológicos históricos y relevantes  obtenidos, el primero contiene todas las estaciones con datos completos y 
el segundo contiene aquellas con datos parciales.

\subsection{Variables bioclimáticas}

A partir de estos ficheros, podemos calcular la biotemperatura media anual (ABT), la precipitación media anual (APP), y, finalmente, 
la relación de evapotranspiración potencial (PER) para cada uno de los puntos geográficos.

Respecto a la biotemperatura, calculamos la temperatura media mensual como la media aritmética de la temperatura máxima y mínima. Despúes,
calculamos la biotemperatura mensual aplicando la regla de Holdridge: si la temperatura media mensual es menor o igual a 0 °C, 
se asigna un valor de 0 °C, si está entre 0 °C y 30 °C, se mantiene el valor original, y si es mayor o igual a 30 °C, se asigna un valor de 30 °C.
Finalmente, la biotemperatura media anual (ABT) se obtiene como la media aritmética de las biotemperaturas anuales, tomadas como la media 
aritmetica de las biotemperaturas mensuales.

El calculo de la precipitación media anual (APP) es más sencillo, ya que se obtiene como la media aritmética de las precipitaciones anuales,
tomadas como la suma de las precipitaciones mensuales. Una vez se tiene la ABT y APP, se calcula la relación de evapotranspiración potencial (PER) 
utilizando la fórmula empírica propuesta por Holdridge:
$PER = \frac{ETP}{APP} = \frac{58.93 \times ABT}{APP}$, donde ETP es la evapotranspiración potencial anual.

Sobre los datos incompletos, hemos decidido interpolar los valores faltantes en las 300 estaciones para las que los datos faltantes son 
únicamente muestras sueltas a lo largo del tiempo de cualquiera de las columnas relevantes para este informe. Las otras 1900 estaciones con datos
incompletos han sido descartadas para el aprendizaje difuso, se utilizarán como muestras para evaluar como el sistema aprendido se comporta
ante datos reales con información incompleta.

Para el proceso de interpolación, se ha utilizado interpolación usando el método de polinomio de interpolación cúbica de hermite por tramos.
Para cada conjunto de muestras agrupadas anualmente, deducimos el valor que se debe escribir en los huecos de las columnas ABT y APP vacíos;
para después calcular el PER a partir de los valores interpolados. El proceso de interpolación no se aplica siempre, si un registro
anual tiene más de 3 meses con datos faltantes, no se realiza la interpolación y se considerea que la estación tiene datos incompletos.

De esta forma, obtendremos datos en los que, si se ha podido interpolar, se tendrán los valores de ABT, APP y PER completos para cada estación.
En otro caso, si sólo no se ha podido iterpolar ABT o APP, se descarta la estación como dato de entrenamiento. Se podrá intentar utilizar para
observar el comportamiento del sistema aprendido ante datos incompletos.

Después de fusionar los datos completos y los datos parciales interpolados, obtenemos un fichero CSV final más de 1000 muestras.

\subsection{Clasificación}

Una vez tenemos las variables bioclimáticas calculadas, podemos proceder a clasificar cada estación en su correspondiente zona de vida de Holdridge.
utilizando nuestro zonificador desarrollado en el Informe 2.

\begin{figure}[H]
    \centering
	\includegraphics[width=\textwidth]{images/zonify_completas.png}
	\caption{Zonificación de estaciones con datos completos}
	\label{fig:zonify_completas}
\end{figure}

La Figura \ref{fig:zonify_completas} muestra la zonificación obtenida para las estaciones con datos completos. Para mejor visualización,
se han expandido los puntos en el mapa para cubrir un área mayor.

\begin{figure}[H]
    \centering
	\includegraphics[width=\textwidth]{images/zonify_parciales.png}
	\caption{Zonificación de estaciones con datos parciales}
	\label{fig:zonify_parciales}
\end{figure}

Véase como en la Figura \ref{fig:zonify_parciales}, los resultados utilizando datos interpolados son lo suficientemente coherentes con los obtenidos a partir de datos completos,
lo que nos permite utilizarlos para el aprendizaje difuso.

\section{Aprendizaje difuso}

\subsection{Configuración}

Para utilizar el software FID3.5 \cite{FID}, hemos tenido que transformar el fichero CSV final en un fichero de texto con el formato
esperado por el software. Para ello, hemos desarrollado un script en Python \texttt{zonify2fid2.4.py} que realiza dicha transformación,
obteniendo el fichero \texttt{.dat} que espera con muestras de entrenamiento. Aprovechando el desarrollo para el Informe 2, el script anterior también procesa la configuración de nuestro motor de inferencia difusa
para generar el fichero de atributos \texttt{.attr} que espera el software FID.

Para aliviar el problema de sobre-representación e infra-representación de ciertas etiquetas en el conjunto de datos, hemos
ajustado el peso de las muestras en función de la frecuencia de aparición de cada etiqueta en el conjunto de datos. La fórmula utilizada
es la siguiente: $p = \sqrt{n}$, donde $p$ es el peso asignado a cada muestra y $n$ es el número de apariciones de la etiqueta 
correspondiente en el conjunto de datos. Además, en el fichero de atributos, hemos eliminado las etiquetas de las zonas de vida que no 
aparecen en el dataset final, para evitar que el software FID genere ramas innecesarias.

TODO RESULTADOS

\clearpage
\section*{Uso de IA}
Durante la elaboración de este informe, se ha utilizado ChatGPT-4 para asistir en la redacción a LaTeX de este documento
mediante la herramienta de OpenAI integrada en Visual Studio Code. También se ha mantenido una conversación 
con Gemini-2.5 de Google para obtener orientación y opiniones sobre la estructura y contenido del informe.
Junto a este documento se entrega un archivo HTML con el historial de la conversación mantenida con la misma.

\clearpage
% Bibliografía
\printbibliography

\end{document}