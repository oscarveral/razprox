% Archivo: informe3.tex

\documentclass[11pt,a4paper]{article}

% Idioma y tipografías
\usepackage[spanish, es-noquoting, es-lcroman, es-tabla]{babel}
\usepackage[T1]{fontenc}
\usepackage[utf8]{inputenc}
\usepackage{lmodern}

% Maquetación y tipografía fina
\usepackage[a4paper,margin=2.5cm]{geometry}
\usepackage{microtype}
\usepackage{setspace}
\onehalfspacing

% Utilidades
\usepackage{csquotes}
\usepackage{graphicx}
\usepackage{xcolor}
\usepackage{booktabs}
\usepackage{siunitx}
\usepackage{amsmath,amssymb}
\usepackage{enumitem}
\usepackage{float}
\setlist{nosep,leftmargin=*,labelsep=0.5em}

% Hipervínculos
\usepackage[hidelinks]{hyperref}
\hypersetup{
    pdftitle={Informe de Aprendizaje Difuso},
    pdfauthor={Juan Diego Gallego Nicolás, Óscar Vera López},
    pdfsubject={Aprendizaje Difuso},
    pdfkeywords={Zona de Vida de Holdridge, Lógica Difusa, Inferencia Difusa}
}

% Bibliografía
\usepackage[
    backend=biber,
    style=ieee,
    sorting=nyt,
    maxbibnames=99
]{biblatex}
\addbibresource{references.bib}

% Subfiguras
\usepackage{subcaption}

% Código (Python)
\usepackage{listings}
\lstset{
	language=Python,
	basicstyle=\ttfamily\small,
	keywordstyle=\color{blue},
	stringstyle=\color{red},
	commentstyle=\color{green!60!black}\itshape,
	numbers=left,
	numberstyle=\tiny,
	stepnumber=1,
	numbersep=5pt,
	showstringspaces=false,
	breaklines=true,
	frame=single,
	captionpos=b
}
% Código JSON
\lstdefinelanguage{JSON}{
    basicstyle=\normalfont\ttfamily,
    numbers=left,
    numberstyle=\tiny,
    stepnumber=1,
    numbersep=8pt,
    showstringspaces=false,
    breaklines=true,
    frame=single,
    literate=
     *{0}{{{\color{blue}0}}}{1}
      {1}{{{\color{blue}1}}}{1}
      {2}{{{\color{blue}2}}}{1}
      {3}{{{\color{blue}3}}}{1}
      {4}{{{\color{blue}4}}}{1}
      {5}{{{\color{blue}5}}}{1}
      {6}{{{\color{blue}6}}}{1}
      {7}{{{\color{blue}7}}}{1}
      {8}{{{\color{blue}8}}}{1}
      {9}{{{\color{blue}9}}}{1}
      {:}{{{\color{red}:}}}{1}
      {,}{{{\color{red},}}}{1}
      {\{}{{{\color{orange}\{}}}{1}
      {\}}{{{\color{orange}\}}}}{1}
      {[}{{{\color{orange}[}}}{1}
      {]}{{{\color{orange}]}}}{1}
      {"}{{{\color{green}"}}}{1}
}


% Comandos útiles

\begin{document}
\begin{titlepage}
    \centering
    % \includegraphics[height=2cm]{logo.png}\par\vspace{1cm} % Descomentar si hay logo
    {\Large Informe Técnico 3}\par\vspace{0.5cm}
    {\huge\bfseries Aprendizaje Difuso\par}\vspace{0.5cm}
    \begin{tabular}{@{}ll@{}}
        Autor 1: & Juan Diego Gallego Nicolás \\
        Contacto 1: & jdiego.gallego@um.es \\
        Autor 2: & Óscar Vera López \\
        Contacto 2: & oscar.veral@um.es \\
        Profesor: & Maria del Carmen Garrido Carrera \\
        Asignatura: & Conocimiento y Razonamiento Aproximado \\
    \end{tabular}
    % Include umu logo
    \vfill
    \includegraphics[height=8cm]{images/SelloUMU-negativo.png}\par\vspace{1cm}
    \vfill
    {\large \today}\par
\end{titlepage}

\pagenumbering{roman}
\clearpage

\tableofcontents
\clearpage
\pagenumbering{arabic}

\section{Introducción}
La clasificación de zonas bioclimáticas
proporciona una base para comprender la distribución de ecosistemas. En este contexto, el Sistema de Zonas de Vida de Holdridge
ofrece un marco robusto para categorizar regiones según sus características climáticas y geográficas.
Concretamente, este sistema utiliza tres magnitudes climáticas principales: la biotemperatura media anual (ABT),
la precipitación media anual (APP) y la relación de evapotranspiración potencial (PER).

\begin{figure}[H]
	\centering
	\includegraphics[width=0.65\textwidth]{images/triangle.png}
	\caption{Diagrama del sistema de Zonas de Vida de Holdridge \cite{WikiHoldridge}}
	\label{fig:holdridge-triangle}
\end{figure}

En la imagen anterior (\ref{fig:holdridge-triangle})
se muestra un diagrama representativo del sistema de Holdridge, donde se visualizan las diferentes zonas de vida en 
función de las tres magnitudes climáticas mencionadas.

\begin{figure}[H]
	\centering
	\includegraphics[width=0.6\textwidth]{images/pisos_altitudinales.png}
	\caption{ABT en función de la latitud y altitud \cite{INRENA1995}}
	\label{fig:abt-latitude}
\end{figure}

Holdridge también propone un método para aproximar la de ABT de un punto geográfico. Para ello,
establece una relación directa entre latitud y la ABT media anual a nivel del mar. A partir de esta relación,
se pueden calcular las desviaciones de la ABT en función de la altitud del punto geográfico con la regla 
empírica: la temperatura disminuye aproximadamente 6 °C por cada 1000 metros de ascenso en altitud (\ref{fig:abt-latitude}).

En este informe abordamos la tarea de aprendizaje difuso a partir de datos que permita la inferencia de la zona de vida de Holdridge correspondiente a un punto geográfico.
Obtendremos datos meteorológicos reales desde utilizando Meteostat \cite{Meteostat}, los cuales procesaremos e introduciremos en el motor de inferencia 
difusa desarrollado para el Informe 2 de la asignatura. A partir de estas inferencias, mediante el
software FIDv3.5 \cite{FID}, podremos construir un árbol de decisión difuso que permita la clasificación automática de zonas de vida de Holdridge.

\section{Preprocesamiento de datos meteorológicos}

En este informe trabajaremos con datos meteorológicos reales obtenidos de la plataforma Meteostat \cite{Meteostat} para la mayoría de países europeos.

\subsection{Obtención de estaciones meteorológicas}

El primer paso ha consistido en la elaboración de un script \texttt{shp2estaciones.py} para
poder recopilar las localizaciones de las estaciones meteorológicas disponibles en Meteostat dado un área geográfica concreta
delimitada por un fichero shapefile \cite{EUShp}. Para ello, hemos utilizado la librería Geopandas de Python para cargar el shapefile
y hemos dividido el área en una malla de puntos equidistantes (en grados de latitud y longitud). Posteriormente, utilizando la API de Meteostat,
seleccionamos la estación meteorológica más cercana a cada punto de la malla. Finalmente, filtramos las estaciones únicas localizadas.
La separación entre los puntos de la malla se ha ido ajustando progresivamente hasta llegar a un total de 0.0625 grados entre puntos (aproximadamente 7 km),
a partir de la cual no se han encontrado nuevas estaciones.

\begin{figure}[H]
    \centering
	\includegraphics[width=\textwidth]{images/estaciones_1.png}
	\caption{Subconjunto cargado desde Meteostat en la Unión Europea}
	\label{fig:total_estaciones}
\end{figure}

El mapa (\ref{fig:total_estaciones}) muestrea el conjunto de puntos seleccionado dentro del shapefile. Un punto en azul indica 
tierra firme, mientras que uno rojo señala la presencia de una estación climática. Mediante este procedimiento se han 
obtenido las localizaciones de 3620 estaciones meteorológicas, las cuales hemos guardado junto a su correspondiente
localización en un fichero CSV.

\subsection{Obtención de datos meteorológicos}

Una vez disponemos de localizacion de las estaciones, mediante un segundo script en Python \texttt{estaciones2datosmet.py} se obtienen,
para cada una de las muestras de estación-coordenadas, los datos meteorológicos históricos con frecuencia mensual desde el año 2015 
a 2020. Durante el proceso se descartan columnas innecesarias para en este informe quedarnos únicamente con las columnas de temperatura mínima (ºC),
temperatura máxima (ºC) y precipitación acumulada (mm).

\label{text:partial}

De las 3620 estaciones obtenidas, 763 de ellas disponen de datos meteorológicos completos para todas las columnas relevantes
en el rango temporal seleccionado (2015-2020). Otras 300 presentan datos parciales tanto para temperatura como para precipitación pero interpolables mediante
imputación por la media (por disponer de datos para ese mes otros años) o interpolación por splines cúbicos (por disponer de datos suficientes en meses anteriores y posteriores).
Las otras 2557 estaciones no tienen datos disponibles, presentan demasiados datos faltantes para ser interpolables o, en el mejor de los casos,
solo disponen de uno de los dos tipos de parámetros. De este último tipo hay unas 1900 estaciones.

\begin{figure}[H]
    \centering
	\includegraphics[width=\textwidth]{images/estaciones_2.png}
	\caption{Completitud de datos meteorológicos históricos por estación}
	\label{fig:parcial_estaciones}
\end{figure}

La figura \ref{fig:parcial_estaciones} muestra la distribución geográfica de las estaciones con datos completos (azul) y datos parciales o ausentes (rojo).
Salta a la vita que la API de Meteostat \cite{Meteostat} no registra estaciones de forma homogénea en Europa, predominando Alemania
como el país con mayor número de estaciones disponibles.

Una vez obtenidos estos datos, se construyen 2 ficheros CSV concatenando las tablas de las localizaciones de las estaciones con
las de datos meteorológicos. El primero contiene todas las estaciones con datos completos y el segundo contiene aquellas con datos parciales.

\subsection{Variables bioclimáticas}

A partir de estos ficheros, podemos calcular la biotemperatura media anual (ABT), la precipitación media anual (APP), y, finalmente, 
la relación de evapotranspiración potencial (PER) para cada uno de los puntos geográficos.

Respecto a la biotemperatura, calculamos la temperatura media mensual como la media aritmética de la temperatura máxima y mínima. Despúes,
calculamos la biotemperatura mensual aplicando la regla de Holdridge: 

\begin{itemize}[leftmargin=4em]
    \item[-] Si la temperatura media mensual es menor o igual a 0 °C, se asigna un valor de 0 °C.
    \item[-] Si la temperatura media mensual está entre 0 °C y 30 °C, se mantiene el valor original.
    \item[-] Si la temperatura media mensual es mayor o igual a 30 °C, se asigna un valor de 30 °C.
\end{itemize}

La biotemperatura media anual (ABT) se obtiene como la media aritmética de las biotemperaturas anuales, tomadas como la media 
aritmetica de las biotemperaturas mensuales.

El cálculo de la precipitación media anual (APP) es más sencillo ya que se obtiene como la suma de las medias mensuales de precipitación.
 
Una vez se tiene la ABT y APP, se calcula la relación de evapotranspiración potencial (PER) 
utilizando la fórmula empírica propuesta por Holdridge:
$PER = \frac{APE}{APP} = \frac{58.93 \times ABT}{APP}$, donde APE es la evapotranspiración potencial anual.

Sobre los datos incompletos, como mencionábamos, hemos realizado el siguiente procedimiento:
\begin{enumerate}[leftmargin=4em]
    \item Si algún par (año, mes) tiene datos faltantes en alguna de sus columnas, el cálculo de la biotemperatura o precipitación acumulada para su mes no tiene en cuenta ese dato.
    \item Si, después de agrupar por mes, algún mes tiene menos de tres datos faltantes en alguna de sus columnas, se imputan los datos faltantes utilizando splines cúbicos por tramos. Para aprovechar la naturaleza cíclica de los datos, se triplican los datos y se concetenan para tomar los datos centrales tras la interpolación.
\end{enumerate}

Las otras 1900 estaciones con datos incompletos (solo tienen ABT o APP) han sido descartadas para el aprendizaje difuso pues no
podemos generar etiquetas suficientemente informadas con nuestro zonificador difuso previo.
Se utilizarán como muestras para evaluar como el sistema aprendido se comporta ante datos reales con información incompleta.

Después de fusionar los datos completos y los datos parciales interpolados, obtenemos un fichero CSV final más de 1000 muestras.

\subsection{Clasificación}

Una vez tenemos las variables bioclimáticas calculadas, podemos proceder a clasificar cada estación en su correspondiente zona de vida de Holdridge
utilizando nuestro zonificador desarrollado en el Informe 2. Este software nos proporciona las tres zonas de vida con mayor grado de pertenencia
y calcula una coloración ponderada en función de las tres zonas de vida obtenidas.

\begin{figure}[H]
    \centering
	\includegraphics[width=\textwidth]{images/zonify_completas.png}
	\caption{Zonificación de estaciones con datos completos.}
	\label{fig:zonify_completas}
\end{figure}

La Figura \ref{fig:zonify_completas} muestra la zonificación obtenida para las estaciones con datos completos.

\begin{figure}[H]
    \centering
	\includegraphics[width=\textwidth]{images/zonify_parciales.png}
	\caption{Zonificación de estaciones con datos interpolados.}
	\label{fig:zonify_parciales}
\end{figure}

Véase como en la Figura \ref{fig:zonify_parciales}, los resultados utilizando datos interpolados son lo suficientemente coherentes con los obtenidos a partir de datos completos,
lo que nos permite utilizarlos para el aprendizaje difuso.

\begin{figure}[H]
    \centering
	\includegraphics[width=\textwidth]{images/zonify_fused.png}
	\caption{Zonificación de estaciones fusionadas.}
	\label{fig:zonify_fused}
\end{figure}

\section{Aprendizaje difuso}

\subsection{Creación del conjunto de datos}

Para utilizar el software FID3.5 \cite{FID}, hemos tenido que transformar el fichero CSV final en un fichero de texto con el formato
esperado por el software. Para ello, hemos desarrollado un script en Python \texttt{zonify2fid3.5.py} que realiza dicha transformación,
obteniendo el fichero \texttt{.dat} que espera con muestras de entrenamiento.

Para aliviar el problema dede desbalanceo en las etiquetas del conjunto de datos, hemos
ajustado el peso de las muestras en función de la frecuencia de aparición de cada etiqueta en el conjunto de datos. La fórmula utilizada
es la siguiente: $p = \frac{1}{\sqrt{n}}$, donde $p$ es el peso asignado a cada muestra y $n$ es el número de apariciones de la etiqueta 
correspondiente en el conjunto de datos. Cabe mencionar que nuestro dataset de entrenamiento, al derivarse únicamente de datos
meteorológicos europeos, sólo contiene 22 de las 38 zonas de vida de Holdridge posibles en las hemos dividido el espacio de salida.
Recuérdese que nuestro zonificador es capaz de distingir entre todas las sub-zonas de vida resultantes de la 
subdivisión según la latitud. Por ejemplo, distinguir entre Bosque Húmedo Templado Frío y Bosque Húmedo Templado Cálido.

\subsection{Atributos y configuración}

Atendiendo a los primeros resultados preeliminares y usando nuestro conocimiento experto, decidimos eliminar la variable de entrada PER, ya que no aporta información adicional relevante
al estar correlacionada con las otras dos variables de entrada (ABT y APP). Incluso, llegaba a perjudicar a la generación
del árbol introduciendo perturbaciones. Sin PER, el árbol generado es más sencillo y con mejor capacidad de generalización.

Hemos explorado dos configuraciones diferentes para el particionado de las variables de entrada (ABT y APP):
\begin{itemize}
    \item Particionado experto: hemos definido las particiones de los conjuntos difusos para cada una de las variables de entrada (ABT, APP)
    utilizando la configuración de nuestro motor de inferencia difusa del Informe 2.
    \item Particionado automático: hemos permitido que el software FID realice un particionado automático de las variables de entrada (ABT, APP),
    ajustando la tolerancia mínima de particionado a 0 y definiendo un rango de entre 7 y 13 conjuntos difusos para cada variable.
\end{itemize}

El fichero de parámetros \texttt{par.template} es una ligera modificación del fichero por defecto. 
Los cambios relevantes son:
\begin{itemize}
    \item f1Build, f2Build, f1Inf y f2Inf se han establecido a la t-norma del producto. Cualquier otra t-norma probada resultaba en un fallo de software.
    \item El método de resolución de conflictos internos (elección de candidato por hoja) se ha establecido a ``best'' para priorizar la elección de la etiqueta con mayor grado de pertenencia.
    \item El método de resolución de conflictos externos (elección de candidato entre hojas) se ha establecido a 2, ya que es el método recomendado para clasificación.
\end{itemize}

Se ha probado a realizar validación cruzada de hasta 10 pliegues obteniendo siempre resultados similares de precisión a los obtenidos sin validación cruzada. Por simplicidad, se han presentado los resultados sin validación cruzada.

\subsection{Resultados}

Inicialmente, utilizando el particionado experto, el árbol obtenido obtiene una precisión mayor al 
80\% validando los datos de entrenamiento. Observando la figura \ref{fig:confusion_experto} se aprecian leves
desviaciones fuera de la diagonal principal en la matriz de confusión, indicando que el sistema
comete algunos errores graves de clasificación. Por ejemplo, la estepa templada fría (ETF), sólo se 
predice correctamente la mitad de las veces, o, bosque boreal muy húmedo tiene una tasa de acierto del 0\%.
El mapa de clasificación resultante (\ref{fig:mapa_experto}) muestra bastante parecido a la división
obtenida con nuestro zonificador.

\begin{figure}[H]
    \centering
	\includegraphics[width=0.6\textwidth]{images/confusion_normal.png}
	\caption{Matriz de confusión con particionado experto}
	\label{fig:confusion_experto}
\end{figure}

\begin{figure}[H]
    \centering
	\includegraphics[width=0.6\textwidth]{images/mapa_normal.png}
	\caption{Mapa de clasificación con particionado experto}
	\label{fig:mapa_experto}
\end{figure}

Para comparar e intentar mejorar resultados, decidimos utilizar el particionado automático.
De esta forma, el software FID es capaz de encontrar divisiones óptimas para cada variable de entrada.
Esta optimización automática ha permitido alcanzar una precisión del 79\% sobre el conjunto de datos de entrenamiento; menor a la obtenida con conocimiento
experto. Observando la matriz de confusión en la figura \ref{fig:confusion_automatico}, se aprecia 
la mitigación de los errores graves de la clasificación anterior. Sin embargo,
la precisión global es ligeramente menor. 
El mapa de clasificación resultante \ref{fig:mapa_automatico} muestra un parecido similar a la zonificación
original obtenida con nuestro zonificador y al mapa obtenido con particionado experto.

\begin{figure}[H]
    \centering
	\includegraphics[width=0.6\textwidth]{images/confusion_auto.png}
	\caption{Matriz de confusión con particionado automático}
	\label{fig:confusion_automatico}
\end{figure}

\begin{figure}[H]
    \centering
	\includegraphics[width=0.6\textwidth]{images/mapa_auto.png}
	\caption{Mapa de clasificación con particionado automático}
	\label{fig:mapa_automatico}
\end{figure}

Finalmente, hemos querido poner a prueba la capacidad del software FID para recrear el sistema de reglas original a partir de datos sintéticos.
En total, creamos 9 muestras representativas para cada zona de vida. Manteniendo el particionado
manual inicial, construimos un árbol utilizando únicamente estos datos. El procedimiento para la generación
ha sido el siguiente (utilizando nuestro particionado experto inicial):
\begin{itemize}
    \item Para cada zona de vida, se obtienen las particiones difusas correspondientes a cada variable de entrada (ABT, APP).
    \item Para cada una de las particiones difusas, se obtienen 3 valores representativos (valor central del intervalo teórico y dos cercanos).
    \item Se generan todas las combinaciones posibles de los valores de cada variable de entrada.
    \item Cada combinación generada se etiqueta con la zona de vida correspondiente.
\end{itemize}
La única excepción en este procedimiento ha sido para las biotemperaturas entre 0 y 1.5 ºC,
que corresponden siempre a la misma zona de vida (desierto polar) y se dispone de tres muestras.

El árbol resultante mostró una precisión mayor al 82\% sobre los datos de 
entrenamiento, ahora tratados únicamente como datos de test. Véase en la figura \ref{fig:confusion_sinteticos} como la matriz de confusión muestra 
menos desviaciones notables fuera de la diagonal principal y como en el mapa de la figura \ref{fig:mapa_sinteticos} la clasificación es ya 
muy parecida la zonificación obtenida con nuestro zonificador.

\begin{figure}[H]
    \centering
	\includegraphics[width=0.6\textwidth]{images/confusion_sint.png}
	\caption{Matriz de confusión con particionado experto y datos sintéticos}
	\label{fig:confusion_sinteticos}
\end{figure}

\begin{figure}[H]
    \centering
	\includegraphics[width=0.6\textwidth]{images/mapa_sint.png}
	\caption{Mapa de clasificación con particionado experto y datos sintéticos}
	\label{fig:mapa_sinteticos}
\end{figure}

A continuación se muetra una tabla resumen con las métricas de precisión, recall y DICE-Score (media entre todas las clases) obtenidas en cada una de las configuraciones exploradas:
\begin{table}[H]
    \centering
    \begin{tabular}{lccc}
        \toprule
        Configuración & Precisión & Recall & DICE-Score \\
        \midrule
        Particionado experto & 0.65 & 0.66 & 0.63 \\
        Particionado automático & 0.72 & 0.58 & 0.60 \\
        Particionado experto con datos sintéticos & 0.78 & 0.77 & 0.76 \\
        \bottomrule
    \end{tabular}
    \caption{Resumen de métricas obtenidas}
    \label{tab:resumen_metricas}
\end{table}

El DICE-Score mide la media armónica entre el accuracy y el recall, por lo que es una métrica
útil para evaluar el rendimiento del sistema en términos de precisión y exhaustividad.
Destáquese como, midiendo el DICE-Score, la configuración con datos sintéticos es 
la que mejor rendimiento obtiene con mucha diferencia.

Hasta ahora hemos evaluado los resultados comparando las predicciones del árbol con la etiqueta de 
mayor pertenencia obtenida con nuestro zonificador. Para complementar los resultados, es interesante
observar cual es la tasa de acierto si tomamos en cuenta tambíen las segundas y terceras etiquetas
de mayor pertenencia.
\begin{itemize}
    \item Particionado experto. Coincidencias totales (1ª, 2ª o 3ª etiqueta): 96.89\%.
    \begin{itemize}
        \item Coincidencias con 1ª etiqueta: 80.32\%.
        \item Coincidencias con 2ª etiqueta: 8.76\%.
        \item Coincidencias con 3ª etiqueta: 7.82\%.
    \end{itemize}
    \item Particionado automático. Coincidencias totales (1ª, 2ª o 3ª etiqueta): 94.63\%.
    \begin{itemize}
        \item Coincidencias con 1ª etiqueta: 79.00\%.
        \item Coincidencias con 2ª etiqueta: 11.39\%.
        \item Coincidencias con 3ª etiqueta: 4.24\%.
    \end{itemize}
    \item Particionado experto con datos sintéticos. Coincidencias totales (1ª, 2ª o 3ª etiqueta): 99.11\%.
    \begin{itemize}
        \item Coincidencias con 1ª etiqueta: 82.67\%.
        \item Coincidencias con 2ª etiqueta: 8.66\%.
        \item Coincidencias con 3ª etiqueta: 7.82\%.
    \end{itemize}
\end{itemize}

Al considerar las segundas y terceras etiquetas de mayor pertenencia,
la tasa de acierto aumenta significativamente en todas las configuraciones exploradas. En la 
configuración con datos sintéticos, llegamos a tener más de un 99\% de acierto. 
Esto evidencia la existencia de zonas de vida transicionales que no se reflejan en el triángulo de clasificación,


\clearpage

\subsection{Reglas generadas}

El árbol generado con particionado experto (\ref{fig:reglas_experto}) realiza una primera
bifurcación basándose en el atributo ABT y después utiliza APP para realizar las hojas finales.
Véase como para la menor biotemperatura posible, dado que tenemos pocos ejemplos en el conjunto
de datos, el árbol no llega a bifurcarse en APP, ya que con la biotemperatura ya es posible
inferir la zona de vida sin tener en cuenta el APP. En total se han generado 34 hojas en el árbol,
12 más que las teóricamente necesarias para cubrir todas las zonas de vida presentes en el conjunto de datos.

\begin{figure}[H]
    \centering
    \includegraphics[width=0.15\textwidth]{images/reglas_experto.png}
    \caption{Árbol de reglas generado con particionado experto}
    \label{fig:reglas_experto}
\end{figure}

En el caso del particionado automático (\ref{fig:reglas_auto}), el árbol generado
cuenta con 112 hojas, lo que lo hace más difícil de interpretar y más propenso al sobreajuste. 

\begin{figure}[H]
    \centering
    \includegraphics[width=0.15\textwidth]{images/reglas_auto.png}
    \caption{Árbol de reglas generado con particionado automático}
    \label{fig:reglas_auto}
\end{figure}

Con los datos sintéticos y particionado experto, ahora el árbol en la figura 
\ref{fig:reglas_sint} contiene reglas para todas las zonas de vida posibles, lo que hace que
se generen todos los caminos posibles en el árbol, justificando la gran precisión obtenida.
Resulta interesante comprobar cómo el número total de hojas es 55, lo que coincide con 
el número de combinaciones de etiquetas de ABT y APP menos uno. Esto indica que el árbol ha aprendido
perfectamente las reglas originales del zonificador difuso.

\begin{figure}[H]
    \centering
    \includegraphics[width=0.15\textwidth]{images/reglas_sint.png}
    \caption{Árbol de reglas generado con particionado experto y datos sintéticos}
    \label{fig:reglas_sint}
\end{figure}

\subsection{Análisis de particiones}

La figura \ref{fig:particion_experta} muestra las particiones definidas manualmente para las variables de entrada,
junto a la función de distribución de dichas variables en el conjunto de datos de entrenamiento. La mayoría de las muestras
se concentran en pocas de particiones. 

\begin{figure}[H]
    \centering
	\includegraphics[width=0.5\textwidth]{images/particion_experta.png}
	\caption{Partición experta de variables de entrada}
	\label{fig:particion_experta}
\end{figure}

Si superponemos la distribución de las muestras sintéticas se aprecia con claridad uniformidad 
con respecto a cada una de las particiones.

\begin{figure}[H]
    \centering
	\includegraphics[width=0.5\textwidth]{images/particion_sint.png}
	\caption{Partición experta con datos sintéticos}
	\label{fig:particion_sint}
\end{figure}

El particionado automático de conjuntos difusos (\ref{fig:particion_auto}), crea particiones más finas allí donde hay mayor densidad de muestras
y particiones más amplias en el caso contrario. Esto permitiría al sistema, en teoría, adaptarse mejor a la distribución real de los datos. No obstante,
como veíamos en el análisis de métricas, este particionado no ha resultado en una mejora del rendimiento del sistema sino todo lo contrario. 
La existencia de particiones muy finas no arregla el problema de las zonas de vida transacionales.

\begin{figure}[H]
    \centering
	\includegraphics[width=0.5\textwidth]{images/particion_auto.png}
	\caption{Partición automática de variables de entrada}
	\label{fig:particion_auto}
\end{figure}

\subsection{Inferencia con datos parciales}

Finalmente, hemos querido evaluar el rendimiento del sistema aprendido
ante datos reales con información incompleta. Para ello, hemos utilizado los datos de las 2927 estaciones de las cuales se dispone el 
ABT y/o el APP. 

\begin{figure}[H]
    \centering
    \includegraphics[width=0.9\textwidth]{images/mapa_all.png}
    \caption{Mapa de clasificación con datos completos y parciales}
    \label{fig:mapa_fused}
\end{figure}

El resultado es el mapa de la figura \ref{fig:mapa_fused}, donde las estaciones
se han clasificado utilizando el árbol generado con particionado experto y datos reales.
El resultado es un mapa coherente con la geografía y clima de Europa, excepto en el territorio alemán,
donde la mayoría de las zonas de vida se clasifican como Desierto Polar. 

\section{Conclusiones}

A pesar de no obtener una precisión perfecta, los árboles generados con el software FID3.5
han demostrado ser capaces de aproximar razonablemente bien el sistema de zonificación difusa. Recordemos que
el sistema de zonificación de Holdridge cuenta con zonas de vida transicionales que no 
se reflejan en el triángulo de clasificación. En consecuencia, tomar por deficiente un clasificador que proporciona una única etiqueta de salida con una precisión
del 80\% no es del todo adecuado. 

Hemos visto que si tenemos en cuentas las segundas y terceras etiquetas de mayor pertenencia la tasa de acierto aumenta significativamente, llegando a más del 96\% en el árbol generado
a partir de datos reales. Esto indica que el sistema es capaz de aproximar bien las zonas de vida transicionales, aunque no pueda representarlas explícitamente en su salida.

El uso de datos sintéticos ha demostrado ser una estrategia efectiva para que el sistema aprenda las reglas originales del zonificador difuso.
Esto nos indica un buen funcionamiento del software y una validación de la metodología empleada.


\clearpage
\section*{Uso de IA}
Durante la elaboración de este informe, se ha utilizado ChatGPT-4 para asistir en la redacción a LaTeX de este documento
mediante la herramienta de OpenAI integrada en Visual Studio Code. También se ha mantenido una conversación 
con Gemini-2.5 de Google para obtener orientación y opiniones sobre la estructura y contenido del informe.
Junto a este documento se entrega un archivo HTML con el historial de la conversación mantenida con la misma.

\clearpage
% Bibliografía
\printbibliography

\end{document}