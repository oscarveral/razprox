% Archivo: informe1.tex

\documentclass[11pt,a4paper]{article}

% Idioma y tipografías
\usepackage[spanish, es-noquoting, es-lcroman, es-tabla]{babel}
\usepackage[T1]{fontenc}
\usepackage[utf8]{inputenc}
\usepackage{lmodern}

% Maquetación y tipografía fina
\usepackage[a4paper,margin=2.5cm]{geometry}
\usepackage{microtype}
\usepackage{setspace}
\onehalfspacing

% Utilidades
\usepackage{csquotes}
\usepackage{graphicx}
\usepackage{xcolor}
\usepackage{booktabs}
\usepackage{siunitx}
\usepackage{amsmath,amssymb}
\usepackage{enumitem}
\usepackage{float}
\setlist{nosep,leftmargin=*,labelsep=0.5em}

% Hipervínculos
\usepackage[hidelinks]{hyperref}
\hypersetup{
    pdftitle={Informe de Zonificación Bioclimática},
    pdfauthor={Juan Diego Gallego Nicolás, Óscar Vera López},
    pdfsubject={Ontología Bioclimática},
    pdfkeywords={Zona de Vida de Holdridge, Lógica Difusa, Inferencia Difusa}
}

% Bibliografía
\usepackage[
    backend=biber,
    style=ieee,
    sorting=nyt,
    maxbibnames=99
]{biblatex}
\addbibresource{references.bib}

% Subfiguras
\usepackage{subcaption}

% Código (Python)
\usepackage{listings}
\lstset{
	language=Python,
	basicstyle=\ttfamily\small,
	keywordstyle=\color{blue},
	stringstyle=\color{red},
	commentstyle=\color{green!60!black}\itshape,
	numbers=left,
	numberstyle=\tiny,
	stepnumber=1,
	numbersep=5pt,
	showstringspaces=false,
	breaklines=true,
	frame=single,
	captionpos=b
}
% Código JSON
\lstdefinelanguage{JSON}{
    basicstyle=\normalfont\ttfamily,
    numbers=left,
    numberstyle=\tiny,
    stepnumber=1,
    numbersep=8pt,
    showstringspaces=false,
    breaklines=true,
    frame=single,
    literate=
     *{0}{{{\color{blue}0}}}{1}
      {1}{{{\color{blue}1}}}{1}
      {2}{{{\color{blue}2}}}{1}
      {3}{{{\color{blue}3}}}{1}
      {4}{{{\color{blue}4}}}{1}
      {5}{{{\color{blue}5}}}{1}
      {6}{{{\color{blue}6}}}{1}
      {7}{{{\color{blue}7}}}{1}
      {8}{{{\color{blue}8}}}{1}
      {9}{{{\color{blue}9}}}{1}
      {:}{{{\color{red}:}}}{1}
      {,}{{{\color{red},}}}{1}
      {\{}{{{\color{orange}\{}}}{1}
      {\}}{{{\color{orange}\}}}}{1}
      {[}{{{\color{orange}[}}}{1}
      {]}{{{\color{orange}]}}}{1}
      {"}{{{\color{green}"}}}{1}
}


% Comandos útiles

\begin{document}
\begin{titlepage}
    \centering
    % \includegraphics[height=2cm]{logo.png}\par\vspace{1cm} % Descomentar si hay logo
    {\Large Informe Técnico 2}\par\vspace{0.5cm}
    {\huge\bfseries Zonificador Bioclimático Difuso\par}\vspace{0.5cm}
    \begin{tabular}{@{}ll@{}}
        Autor 1: & Juan Diego Gallego Nicolás \\
        Contacto 1: & jdiego.gallego@um.es \\
        Autor 2: & Óscar Vera López \\
        Contacto 2: & oscar.veral@um.es \\
        Profesor: & Mercedes Valdés Vela \\
        Asignatura: & Conocimiento y Razonamiento Aproximado \\
    \end{tabular}
    % Include umu logo
    \vfill
    \includegraphics[height=8cm]{images/SelloUMU-negativo.png}\par\vspace{1cm}
    \vfill
    {\large \today}\par
\end{titlepage}

\pagenumbering{roman}
\clearpage

\tableofcontents
\clearpage
\pagenumbering{arabic}

\section{Introducción}
La clasificación de zonas bioclimáticas
proporciona una base para comprender la distribución de ecosistemas. En este contexto, el Sistema de Zonas de Vida de Holdridge
ofrece un marco robusto para categorizar regiones según sus características climáticas y geográficas.
Concretamente, este sistema utiliza tres magnitudes climáticas principales: la biotemperatura media anual (ABT),
la precipitación media anual (APP) y la relación de evapotranspiración potencial (PER). En la siguiente imagen (\ref{fig:holdridge-triangle})
se muestra un diagrama representativo del sistema de Holdridge, donde se visualizan las diferentes zonas de vida en función de las tres magnitudes climáticas mencionadas.

\begin{figure}[H]
	\centering
	\includegraphics[width=0.65\textwidth]{images/triangle.png}
	\caption{Diagrama del sistema de Zonas de Vida de Holdridge \cite{WikiHoldridge}}
	\label{fig:holdridge-triangle}
\end{figure}

Holdridge también propone un método para aproximar la de ABT de un punto geográfico. Para ello,
establece una relación directa entre latitud y la ABT media anual a nivel del mar. A partir de esta relación,
se pueden calcular las desviaciones de la ABT en función de la altitud del punto geográfico con la regla 
empírica: la temperatura disminuye aproximadamente 6 °C por cada 1000 metros de ascenso en altitud (\ref{fig:abt-latitude}).

\begin{figure}[H]
	\centering
	\includegraphics[width=0.6\textwidth]{images/pisos_altitudinales.png}
	\caption{ABT en función de la latitud y altitud \cite{INRENA1995}}
	\label{fig:abt-latitude}
\end{figure}

En este informe abordamos el desarrollo de un sistema de zonificación bioclimática basado en lógica difusa,
que permita clasificar regiones geográficas según el sistema de Holdridge. En particular, hemos implementado
dos sistemas de inferencia difusa: uno que permite inferir la biotemperatura media anual de una región a 
partir de su latitud y altitud, y otro que clasifica la zona de vida bioclimática en función de la ABT, APP y PER.
Para ello, hemos desarrollado una implementación en Python partiendo desde cero, sin utilizar librerías específicas 
de lógica difusa o sistemas de inferencia difusa para maximizar la comprensión del proceso.

\section{Metodología}

\subsection{Paquete bioclas}
Hemos desarrollado un paquete en Python llamado \texttt{bioclas} con un subpaquete \texttt{fuzzylogic} que implementa los sistemas de inferencia difusa. 
El código está disponible en el repositorio de GitHub: \url{https://github.com/oscarveral/razprox}. 

El paquete \texttt{fuzzylogic} proporciona herramientas para definir conjuntos difusos con cualquier función de membresía,
variables lingüísticas, operadores difusos, reglas difusas y sistemas de inferencia difusa. 
La base de la lógica difusa, el conjunto difuso, se implementa en la clase \texttt{FuzzySet}, que permite definir conjuntos difusos con funciones de membresía
a partir de expresiones lambda. Además, proporcionamos funciones de pertenencia comunes como triangular, trapezoidal,
sigmoidal, en forma de 'S' y en forma de '$\Pi$'. También se implementan métodos para el cálculo de propiedades como la altura, el núcleo, el soporte... 

A partir de los conjuntos difusos se definen variables lingüísticas mediante la clase \texttt{FuzzyVariable}, que agrupa varios conjuntos difusos bajo un mismo nombre.
Los operadores difusos (AND, OR, NOT) se implementan como funciones que operan sobre los conjunto difusos tomando su función de membresía y devolviendo un nuevo conjunto difuso resultante.
Para poder trabajar con t-normas y t-conormas conjugadas, hemos utilizado una factoría abstracta y una clase \texttt{FuzzyOperationsSet} que permite definir diferentes familias de operadores difusos:
\textit{minimo/máximo}, \textit{algebraica}, \textit{drástica}, \textit{dubois-prade}, \textit{yager} y \textit{schweizer-sklar}.

Finalmente, la inferencia difusa es posible gracias a las clases \texttt{FuzzyRule} y \texttt{FIS}.
Una regla difusa en nuestro sistema aparece siempre en la forma \textit{IF a is A AND b is B antecedentes ... THEN x is X}, donde las condiciones del \textit{IF} se representan como una lista de tuplas (variable lingüística, conjunto difuso)
y la conclusión del \textit{THEN} como una tupla (variable lingüística, conjunto difuso). El sistema de inferencia difusa \texttt{FIS} agrupa varias reglas difusas y permite realizar inferencias difusas mediante los métodos de \textit{Mandami} y \textit{Larsen}.

\clearpage

\subsection{Definición de variables lingüísticas}

Para la definición de las variables lingüísticas y sus conjuntos difusos hemos utilizado un formato de archivo JSON.
Cada variable lingüística se define como una clave (nombre de la variable), su tipo (cuantitativa o cualitativa) y una lista de etiquetas asociadas.
Cada etiqueta contiene el nombre del conjunto difuso y un atributo que depende del tipo de la variable. 

En el caso de las variables cuantitativas, se proporcionan los parámetros de dominio (intervalo en el que está definida la variable) y
la escala (lineal o logarítmica). El uso de distintos tipos de escalas se sustenta en el hecho de que las magnitudes climáticas que utiliza el 
sistema de Holdridge (\ref{fig:holdridge-triangle}) aparecen en esta distribución, por lo que es conveniente definir sus funciones de membresía en consecuencia.
Las etiquetas de las variables cuantitativas tienen asociado el intervalo de soporte del conjunto difuso. Podemos ver un ejemplo:

\begin{lstlisting}[language=JSON, caption={Definición de variable lingüística cuantitativa ABT}, label={lst:json-app}]
"ABT": {
        "Tipo": "Cuantitativa",
        "Dominio": [-1.0, 5.33],
        "Escala": {
            "Tipo": "Logaritmica",
            "Base": 2.0,
            "Constante": 0.75
        },
        "Etiquetas": {
            "0a1.5": [-1.0, 1.0],
            "1.5a3": [1.0, 2.0],
            ...
\end{lstlisting}

Por otro lado, las variables cualitativas no requieren parámetros de dominio ni escala. Como la única variable
cualitativa que hemos definido es la zona de vida bioclimática, las etiquetas contienen el nombre de la zona de vida y
se asocian a una tripla RGB para su representación gráfica. Un ejemplo se muestra a continuación:

\begin{lstlisting}[language=JSON, caption={Definición de variable lingüística cualitativa Zona de Vida}, label={lst:json-zona-vida}]
"ZonaDeVida": {
        "Tipo": "Cualitativa",
        "Etiquetas": {
            "Desierto-Polar": [224, 224, 248],
            "Tundra-Seca": [217, 217, 242],
            ...
\end{lstlisting}

\clearpage

\subsubsection{ABT}

La biotemperatura media anual (ABT) se define en el intervalo \([-1, 5.33]\) que corresponde a \([0.375,30]\) °C en escala lineal.
La representación del cero en esta es imposible, por lo que valores inferiores a 0.375 °C se representan como -1.
La conversión entre ambas escalas se realiza mediante las fórmulas:
$$
    ABT_{log} = \log_{2}\left(\frac{ABT_{lin}}{0.75}\right), ABT_{lin} = 0.75 \cdot 2^{ABT_{log}}
$$

\begin{table}[H]
\centering
\caption{Soportes de las etiquetas de ABT en escala logarítmica y lineal}
\label{tab:abt-soportes}
\begin{tabular}{@{}lll@{}}
\toprule
Etiqueta & Soporte logarítmico & Soporte lineal (°C) \\
\midrule
0a1.5 & [-1.0, 1.0] & [0.375, 1.5] \\
1.5a3 & [1.0, 2.0] & [1.5, 3.0] \\
3a6 & [2.0, 3.0] & [3.0, 6.0] \\
6a12 & [3.0, 4.0] & [6.0, 12.0] \\
12a18 & [4.0, 4.5] & [12.0, 18.0] \\
18a24 & [4.5, 5.0] & [18.0, 24.0] \\
24a30 & [5.0, 5.33] & [24.0, 30.0] \\
\bottomrule
\end{tabular}
\end{table}

\subsubsection{APP}

La precipitación media anual (APP) se define en el intervalo \([-4.0, 4.0]\) que corresponde a \([62.5, 16000]\) mm en escala lineal.
La conversión entre ambas escalas se realiza mediante las fórmulas:
$$
    APP_{log} = \log_{2}\left(\frac{APP_{lin}}{1000}\right), APP_{lin} = 1000 \cdot 2^{APP_{log}}
$$
\begin{table}[H]
\centering
\caption{Soportes de las etiquetas de APP en escala logarítmica y lineal}
\label{tab:app-soportes}
\begin{tabular}{@{}lll@{}}
\toprule
Etiqueta & Soporte logarítmico & Soporte lineal (mm) \\
\midrule
62.5a125 & [-4.0, -3.0] & [62.5, 125] \\
125a250 & [-3.0, -2.0] & [125, 250] \\
250a500 & [-2.0, -1.0] & [250, 500] \\
500a1000 & [-1.0, 0.0] & [500, 1000] \\
1000a2000 & [0.0, 1.0] & [1000, 2000] \\
2000a4000 & [1.0, 2.0] & [2000, 4000] \\
4000a8000 & [2.0, 3.0] & [4000, 8000] \\
8000a16000 & [3.0, 4.0] & [8000, 16000] \\
\bottomrule
\end{tabular}
\end{table}

\subsubsection{PER}
La relación de evapotranspiración potencial (PER) se define en el intervalo \([-4.0, 4.0]\) en escala logarítmica y \([0.125, 32.0]\) en escala lineal.
La conversión entre ambas escalas se realiza mediante las fórmulas:
$$
    PER_{log} = \log_{2}\left(\frac{PER_{lin}}{2}\right), PER_{lin} = 2 \cdot 2^{PER_{log}}
$$
\begin{table}[H]
\centering
\caption{Soportes de las etiquetas de PER en escala logarítmica y lineal}
\label{tab:per-soportes}
\begin{tabular}{@{}lll@{}}
\toprule
Etiqueta & Soporte logarítmico & Soporte lineal \\
\midrule
0.125a0.25 & [-4.0, -3.0] & [0.125, 0.25] \\
0.25a0.5 & [-3.0, -2.0] & [0.25, 0.5] \\
0.5a1 & [-2.0, -1.0] & [0.5, 1.0] \\
1a2 & [-1.0, 0.0] & [1.0, 2.0] \\
2a4 & [0.0, 1.0] & [2.0, 4.0] \\
4a8 & [1.0, 2.0] & [4.0, 8.0] \\
8a16 & [2.0, 3.0] & [8.0, 16.0] \\
16a32 & [3.0, 4.0] & [16.0, 32.0] \\
\bottomrule
\end{tabular}
\end{table}

\subsubsection{Latitud}

La latitud se define de forma lineal en el intervalo \([0, 90]\) grados.

\begin{table}[H]
\centering
\caption{Soportes de las etiquetas de Latitud}
\label{tab:latitud-soportes}
\begin{tabular}{@{}ll@{}}
\toprule
Etiqueta & Soporte (grados) \\
\midrule
Tropical & [0, 13] \\
Subtropical & [13, 27.5] \\
Templado-Cálido & [27.5, 42] \\
Templado-Frío & [42, 56.5] \\
Boreal & [56.5, 63.75] \\
Sub-Polar & [63.75, 68] \\
Polar & [68, 90] \\
\bottomrule
\end{tabular}
\end{table}

\clearpage

\subsubsection{Altitud}
La altitud se define de forma lineal en el intervalo \([0, 5000]\) metros.
\begin{table}[H]
\centering
\caption{Soportes de las etiquetas de Altitud}
\label{tab:altitud-soportes}
\begin{tabular}{@{}ll@{}}
\toprule
Etiqueta & Soporte (metros) \\
\midrule
0a1000 & [0, 1000] \\
1000a2000 & [1000, 2000] \\
2000a3000 & [2000, 3000] \\
3000a4000 & [3000, 4000] \\
4000a4500 & [4000, 4500] \\
4500a4750 & [4500, 4750] \\
4750a5000 & [4750, 5000] \\
\bottomrule
\end{tabular}
\end{table}

\subsection{Funciones de pertenencia}

Hemos optado por definir las funciones de pertenencia de los conjuntos difusos como funciones en forma de $\Pi$,
ya que permiten una transición suave entre conjuntos difusos adyacentes y evitan discontinuidades bruscas en la inferencia.
Para los conjuntos extremos, las funciones de pertenencia se muestran degeneradas en forma de 'S' para mantener la normalidad
de la variable. En todos los casos, el máximo de la función de pertenencia se encuentra en el centro del soporte del conjunto difuso.
% Grid 3x2 de figuras con las funciones de pertenencia de las variables lingüísticas
\begin{figure}[H]
    \centering
    \begin{subfigure}[b]{0.32\textwidth}
        \centering
        \includegraphics[width=\textwidth]{images/ABT_membership_functions.png}
        \caption{Funciones de pertenencia de ABT}
        \label{fig:funciones_abt}
    \end{subfigure}
    \hfill
    \begin{subfigure}[b]{0.32\textwidth}
        \centering
        \includegraphics[width=\textwidth]{images/APP_membership_functions.png}
        \caption{Funciones de pertenencia de APP}
        \label{fig:funciones_app}
    \end{subfigure}
    \hfill
    \begin{subfigure}[b]{0.32\textwidth}
        \centering
        \includegraphics[width=\textwidth]{images/PER_membership_functions.png}
        \caption{Funciones de pertenencia de PER}
        \label{fig:funciones_per}
    \end{subfigure}
    \vskip\baselineskip
    \begin{subfigure}[b]{0.32\textwidth}
        \centering
        \includegraphics[width=\textwidth]{images/Latitud_membership_functions.png}
        \caption{Funciones de pertenencia de Latitud}
        \label{fig:funciones_latitud}
    \end{subfigure}
    \begin{subfigure}[b]{0.32\textwidth}
        \centering
        \includegraphics[width=\textwidth]{images/Altitud_membership_functions.png}
        \caption{Funciones de pertenencia de Altitud}
        \label{fig:funciones_altitud}
    \end{subfigure}
    \caption{Funciones de pertenencia de las variables lingüísticas}
    \label{fig:funciones_pertenencia}
\end{figure}

\subsection{Definición de los sistemas de inferencia difusa}

De forma similar a las variables lingüísticas, los sistemas de inferencia difusa se definen mediante archivos JSON. 
Cada SID se define a partir de una lista de antecedentes, un consecuente y un conjunto de reglas. Antes de la carga del SID, las
variables lingüísticas y los conjuntos difusos asociados deberán estar definidos en el sistema ya que estos ficheros
no definen sus rangos o etiquetas. 

\begin{lstlisting}[language=JSON, caption={Definición de un sistema de inferencia difusa}, label={lst:json-fis}]
"a_variables": ["Latitud", "Altitud"],
"c_variable": "ABT",
"rules": {
    "ruleTropical1":
    {
        "antecedentes": {
            "Latitud": "Tropical",
            "Altitud": "0a1000"
        },
        "consecuente": {
            "ABT": "24a30"
        }
    }...
\end{lstlisting}

Los mecanismos de inferencia no se definen en el fichero JSON, sino que se especifican al cargar el sistema de inferencia difusa en Python.

\subsubsection{Definición de FIS-Biotem}

El primer sistema de inferencia difusa implementado es el FIS-Biotem, que permite inferir la biotemperatura media anual (ABT)
de un punto geográfico a partir de su latitud y altitud. Este sistema contiene un total de 42 reglas difusas que cubren 
todas las combinaciones posibles de regiones. Para elaborarlas nos hemos basado en la división en pisos altudinales de la 
figura \ref{fig:abt-latitude}. No obstante, por simplificar el sistema y no tener que definir funciones de pertenencia
adaptadas a triángulos del plano latitud-altitud, hemos optado por ``retangularizar'' las regiones. Como veremos más adelante,
esta simplificación no afecta significativamente a la precisión del sistema con la debida elección del método de inferencia y defuzzificación.


\begin{lstlisting}[language=JSON, caption={Definición del sistema de inferencia difusa FIS-Biotem}, label={lst:json-fis}]
"a_variables": ["Latitud", "Altitud"],
"c_variable": "ABT",
"rules": {
    "ruleTropical1":
    {
        "antecedentes": {
            "Latitud": "Tropical",
            "Altitud": "0a1000"
        },
        "consecuente": {
            "ABT": "24a30"
        }
    }...
\end{lstlisting}

\subsubsection{Definición FIS-Zonify}

El segundo sistema de inferencia difusa implementado es el FIS-Zonify, que permite clasificar la zona de vida bioclimática
de un punto geográfico a partir de su biotemperatura media anual (ABT), precipitación media anual (APP) y relación de evapotranspiración potencial (PER).
Este sistema contine un total de 38 reglas difusas que cubren todas las zonas de vida definidas en el sistema de Holdridge. 
Como se ve en el siguiente fragmento de código \ref{lst:json-fis-zonify}, se permite que las reglas tengan un número variable de antecedentes,
de modo que algunas zonas de vida se definen únicamente en función de una o dos magnitudes climáticas.

\begin{lstlisting}[language=JSON, caption={Definición del sistema de inferencia difusa FIS-Zonify}, label={lst:json-fis-zonify}] 
"a_variables": ["ABT", "APP", "PER"],
"c_variable": "ZonaDeVida",
"rules": {
    "rule0": {
        "antecedentes": {
            "ABT": "0a1.5"
        },
        "consecuente": {
            "ZonaDeVida": "Desierto-Polar"
        }
    },

    "rule1_1": {
        "antecedentes": {
            "ABT": "1.5a3",
            "APP": "62.5a125",
            "PER": "1a2"
        },
        "consecuente": {
            "ZonaDeVida": "Tundra-Seca"
        }
    }...
\end{lstlisting}

\clearpage

\section{FIS-Biotem: Resultados y Evaluación}

Para utilizar el fichero de configuración del FIS-Biotem (\ref{lst:json-fis}), hemos desarrollado un script en Python \texttt{biotem.py}
que carga las variables lingüísticas y el sistema de inferencia difusa, y permite realizar inferencias a partir de latitud y altitud.
El script acepta como entrada un fichero CSV con las coordenadas geográficas (latitud, altitud) y el APP asociado a cada punto. 
Genera como salida un fichero CSV con la biotemperatura media anual (ABT) inferida para cada punto y con la PER calculada a partir de la ABT y APP;
un fichero de texto con un resumen de los resultados y un plot que muestra mediante un gradiente de color la distribución de la ABT en la región estudiada.
Se puede especificar el método de inferencia (Mandami o Larsen) y el método de defuzzificación (Centroide o Media de Máximos) mediante argumentos en la línea de comandos.
La t-norma y t-conorma utilizadas en los operadores difusos se derivan del método de inferencia seleccionado (Mandami: mínimo/máximo, Larsen: algebraica).

\subsection{Evaluación del FIS-Biotem}

Para evaluar el rendimiento del FIS-Biotem, hemos utilizado un conjunto de datos sintéticos para cubrir todo el espacio latitud-altitud.
En total, se han generado 4186 puntos de prueba distribuidos uniformemente en latitud (0° a 90°) y altitud (0 a 5000 m).
A continuación, utilizamos el script para obtener la imagen de la distribución de la ABT inferida \ref{fig:biotem-result}.
\begin{figure}[H]
    \centering
    \begin{subfigure}[b]{0.35\textwidth}
        \centering
        \includegraphics[width=\textwidth]{images/biotem_mandami_centroid.png}
        \caption{Mandami + Centroide}
        \label{fig:biotem-mandami-centroid}
    \end{subfigure}
    \begin{subfigure}[b]{0.35\textwidth}
        \centering
        \includegraphics[width=\textwidth]{images/biotem_mandami_mom.png}
        \caption{Mandami + Media de Máximos}
        \label{fig:biotem-mandami-mom}
    \end{subfigure}
    \vskip\baselineskip
    \begin{subfigure}[b]{0.35\textwidth}
        \centering
        \includegraphics[width=\textwidth]{images/biotem_larsen_centroid.png}
        \caption{Larsen + Centroide}
        \label{fig:biotem-larsen-centroid}
    \end{subfigure}
    \begin{subfigure}[b]{0.35\textwidth}
        \centering
        \includegraphics[width=\textwidth]{images/biotem_larsen_mom.png}
        \caption{Larsen + Media de Máximos}
        \label{fig:biotem-larsen-mom}
    \end{subfigure}
    \caption{Distribución de la ABT inferida por el FIS-Biotem con diferentes métodos de inferencia y defuzzificación}
    \label{fig:biotem-result}
\end{figure}

Como se observa en las imágenes, los dos métodos de inferencia (Mandami y Larsen) dan resultados similares. 
El método de defuzzificación por centroide proporciona una transición muy suave entre regiones, llegando casi 
a desaparecer esa sensación de división en rectángulos que se había introducido al definir las reglas.
Por otro lado, el método de la media de máximos genera transiciones discontinuas, lo cual tiene sentido dada la naturaleza
de las funciones de pertenencia utilizadas: al ser todas en forma de $\Pi$, los máximos de los conjuntos difusos se encuentran cerca de su centro,
por lo que la métrica tiende a mantenerse constante hasta que hay un cambio significativo en la activación de los conjuntos difusos.

Para cuantificar el error del sistema, hemos comparado las cuatro configuraciones anteriores con una función de referencia
basada en la regla empírica de Holdridge que relaciona latitud, altitud y ABT (\ref{fig:abt-latitude}):
$$
    ABT_{basal} = \max \left\{0, 30-28.5 \cdot \frac{Latitud}{68}\right\}, ABT = ABT_{basal} - 6 \cdot \frac{Altitud}{1000}
$$
Calculamos el error absoluto medio (MAE), el error cuadrático medio (RMSE) y el error máximo para cada configuración, obteniendo los resultados
resumidos en la tabla \ref{tab:biotem-eval}.
\begin{table}[H]
\centering
\caption{Evaluación del FIS-Biotem con diferentes métodos de inferencia y defuzzificación}
\label{tab:biotem-eval}
\begin{tabular}{@{}lccc@{}}
\toprule
Configuración & MAE (°C) & RMSE (°C) & Error Máximo (°C) \\
\midrule
Mandami + Centroide & 1.42 & 1.73 & 4.15 \\
Mandami + Media de Máximos & 2.23 & 2.69 & 7.41 \\
Larsen + Centroide & 1.56 & 1.87 & 4.19 \\
Larsen + Media de Máximos & 2.18 & 2.61 & 6.80 \\
\bottomrule
\end{tabular}
\end{table}

\begin{figure}[H]
    \centering
    \begin{subfigure}[b]{0.45\textwidth}
        \centering
        \includegraphics[width=\textwidth]{images/error_mc.png}
        \caption{Mandami + Centroide}
        \label{fig:biotem-mandami-centroid}
    \end{subfigure}
    \hfill
    \begin{subfigure}[b]{0.45\textwidth}
        \centering
        \includegraphics[width=\textwidth]{images/error_mm.png}
        \caption{Mandami + Media de Máximos}
        \label{fig:biotem-mandami-mom}
    \end{subfigure}
    \vskip\baselineskip
    \begin{subfigure}[b]{0.45\textwidth}
        \centering
        \includegraphics[width=\textwidth]{images/error_lc.png}
        \caption{Larsen + Centroide}
        \label{fig:biotem-larsen-centroid}
    \end{subfigure}
    \hfill
    \begin{subfigure}[b]{0.45\textwidth}
        \centering
        \includegraphics[width=\textwidth]{images/error_lm.png}
        \caption{Larsen + Media de Máximos}
        \label{fig:biotem-larsen-mom}
    \end{subfigure}
    \caption{ABT calculada - ABT predicha FIS-Biotem con diferentes métodos de inferencia y defuzzificación}
    \label{fig:biotem-result}
\end{figure}

Las métricas de error confirman las observaciones previas: el método de defuzzificación por centroide ofrece un mejor rendimiento
en comparación con la media de máximos, independientemente del método de inferencia utilizado. 
Además, el método de Mandami proporciona un rendimiento ligeramente superior al de Larsen.

Apreciamos también cómo los mayores errores se concentran en los extremos del dominio. Esto puede deberse a la elección
de las funciones de pertenencia: el intervalo 27-30º tiene una pertenencia de 1 al conjunto difuso 24a30, por lo que la 
defuzzificación del centroide nunca podrá superar el valor 28,5ºC. De todas formas, las regiones españolas peninsulares
e insulares se encuentran en el rango 27-42º de latitud, por lo que este error no afecta a la aplicación práctica del sistema en nuestro país.

\begin{figure}[H]
    \centering
    \begin{subfigure}[b]{0.45\textwidth}
        \centering
        \includegraphics[width=\textwidth]{images/defuzz1.png}
        \caption{Función de pertenencia agregada 1}
    \end{subfigure}
    \hfill
    \begin{subfigure}[b]{0.45\textwidth}
        \centering
        \includegraphics[width=\textwidth]{images/defuzz2.png}
        \caption{Función de pertenencia agregada 2}
    \end{subfigure}
    \caption{Ejemplos de funciones de pertenencia agregadas que muestran limitaciones en la defuzzificación por media de máximos.}
    \vskip\baselineskip
\end{figure}

\subsection{Aplicación del FIS-Biotem a datos reales}

Para probar el FIS-Biotem con datos reales, hemos utilizado un conjunto de datos geoclimáticos del territorio español, muestreados cada diez kilómetros proporcionados para la práctica
\texttt{DATOS-CUADRICULA.csv}. Con estos datos, hemos ejecutado el script \texttt{biotem.py} para obtener la distribución de la ABT en la región estudiada.
Graficando los resultados obtenidos con el método de Mandami y defuzzificación por centroide, obtenemos la imagen \ref{fig:biotem-real-data}.
\begin{figure}[H]
    \centering
    \includegraphics[width=0.6\textwidth]{images/biotem_real_data.png}
    \caption{Distribución de la ABT inferida por el FIS-Biotem con datos de España}
    \label{fig:biotem-real-data}
\end{figure}
Los resultados obtenidos son coherentes con la geografía española. Las zonas costeras y las islas tienen transiciones suaves con respecto a la latitud. 
Los accidentes geográficos, como los pirineos, la cordellera bética, la meseta central o el Teide en Tenerife se reflejan claramente en la distribución de la ABT
como zonas de menor biotemperatura debido a su altitud.

\clearpage

\section{FIS-Zonify: Resultados y coloreado de zonas de vida}

Para utilizar el fichero de configuración del FIS-Zonify, hemos desarrollado un script en Python \texttt{zonify.py}
que carga las variables lingüísticas y el sistema de inferencia difusa, y permite realizar inferencias a partir de ABT, APP y PER.
Idealmente, el script acepta como entrada un archivo CSV formateado por FIS-Biotem, pero también puede aceptar un fichero con las tres magnitudes climáticas directamente.
Genera como salida un otro archivo CSV en el que se indica para cada punto las tres zonas de vida con mayor grado de pertenencia y los valores de los 
canales RGB asociados; un fichero de texto con un resumen de los resultados y un plot que muestra la distribución de las zonas de vida en la región estudiada.
El coloreado se realiza mediante una suma de los colores RGB de las zonas de vida ponderados por su grado de pertenencia (normalizándolos previamente).

En el caso de este script solo se permite indicar el método de inferencia (Mandami o Larsen), utilizando siempre la defuzzificación del color 
descrita anteriormente. 

\begin{figure}[H]
    \centering
    \includegraphics[width=0.5\textwidth]{images/evaluation_antecedent.png}
    \caption{Grado de satisfacibilidad de los antecedentes de una regla difusa en función de los valores de ABT, APP y PER}
    \label{fig:zonify-color-mixing}
\end{figure}

\subsection{Aplicación del FIS-Zonify a datos de FIS-Biotem}

Para probar el FIS-Zonify, hemos utilizado los resultados obtenidos previamente con el FIS-Biotem para los datos geoclimáticos del territorio español.
Hemos ejecutado el script \texttt{zonify.py} para obtener la distribución de las zonas de vida en la región estudiada.
Graficando los resultados obtenidos con el método de Mandami, obtenemos la imagen \ref{fig:zonify-real-data}.
\begin{figure}[H]
    \centering
    \includegraphics[width=1.0\textwidth]{images/zonify_peninsula.png}
    \caption{Distribución de las zonas de vida inferida por el FIS-Zonify con datos de España}
    \label{fig:zonify-real-data}
\end{figure}

\subsection{Aplicación del FIS-Zonify a datos de estaciones climáticas}

Hemos utilizado FIS-Zonify también sobre el conjunto de datos de estaciones climáticas españolas de los
recursos de la asignatura. La baja densidad de puntos hace que la representación gráfica
no sea tan detallada como en el caso anterior, por lo que hemos desarrollado un script adicional \texttt{color\_map.py}
que genera una imagen con una resolución de 0.01 grados latiudinales $\times$ 0.01 grados altitudinales (aproximadamente 1 km $\times$ 1 km)
utilizando interpolación a partir del vecino más cercano. Además, aplicamos una máscara para mostrar únicamente las zonas terrestres.

\begin{figure}[H]
    \centering
    \includegraphics[width=0.7\textwidth]{images/color_estaciones.png}
    \caption{Distribución de las zonas de vida inferida por el FIS-Zonify con datos de estaciones climáticas españolas. Las estaciones de Tenerife y Murcia no tienen datos completos y se muestran en azul.}
    \label{fig:zonify-estaciones}
\end{figure}

La diferencia de clasificación entre las dos imágenes es evidente. En el caso de la biotemperatura inferida por FIS-Biotem,
los colores rojos son predominantes, correspondientes a zonas de vida frías y templadas. En cambio, en la imagen generada a partir de las estaciones climáticas
sugiere una la presencia de climas tropicales y templados cálidos.

\subsection{Comparación de métodos}

Para realizar una comparación entre la clasificación dada por los dos conjuntos de datos, vamos a comparar las zonas de vida principales
obtenidas en cada punto de las estaciones climáticas.

\begin{table}[H]
\centering
\caption{Zonas de vida dadas por la zona geográfica y por las estaciones climáticas}
\label{tab:zonas-comparacion}
\begin{tabular}{@{}lccccc@{}}
\toprule
Estación & Provincia & Zona Vida (FIS-Biotem) & Zona Vida (Estaciones) \\
\midrule
BU02 & Burgos & Bosque-Húmedo-Templado-Frío & Bosque-Seco-Templado-Cálido \\
LU02 & Lugo & Bosque-Pluvial-Templado-Húmedo & Bosque-Seco-Templado-Cálido \\
M01 & Madrid & Bosque-Húmedo-Templado-Frío & Matorral-Espinoso-Templado-Cálido \\
A12 & Alicante & Bosque-Húmedo-Templado-Frío & Floresta-Espinosa-Subtropical \\
\bottomrule
\end{tabular}
\end{table}

Claramente, las zonas de vida inferidas a partir de las estaciones climáticas tienden a ser más cálidas y secas que las obtenidas
a partir de los datos geoclimáticos. Esto puede deberse a varios factores, como el hecho de que nuestro
FIS-Biotem utiliza datos de Perú o al cambio climático que ha afectado a España en las últimas décadas.

\section{Conclusiones}

Aunque no todo el material ha sido necesario para la práctica, el paquete desarrollado \texttt{bioclas.fuzzylogic} proporciona una base sólida para la implementación de sistemas de inferencia difusa.
Hemos podido definir dos sistemas de inferencia difusa completos (FIS-Biotem y FIS-Zonify) y evaluar su rendimiento utilizando datos sintéticos y reales.
Aunque FIS-Biotem ha demostrado un buen rendimiento en la inferencia de la biotemperatura media anual (ABT) con con respecto a la regla empírica, 
la existencia de esta regla hace que sea en la práctica innecesario el uso de un sistema difuso para este propósito. No obstante, FIS-zonify
sí se demuestra útil para para solventar el problema de las zonas de vida transicionales del sistema de Holdridge, proporcionando una clasificación más matizada.

\clearpage
\section*{Uso de IA}
Durante la elaboración de este informe, se ha utilizado ChatGPT-4 para asistir en la redacción a LaTeX de este documento
mediante la herramienta de OpenAI integrada en Visual Studio Code. También se ha mantenido una conversación 
con Gemini-2.5 de Google para obtener orientación y opiniones sobre la estructura y contenido del informe.
Junto a este documento se entrega un archivo HTML con el historial de la conversación mantenida con la misma.

\clearpage
% Bibliografía
\printbibliography

\clearpage
\appendix
\section*{Mapa de clasificación por zona geográfica}
\begin{figure}[H]
    \centering
    \includegraphics[width=1.0\textwidth]{images/color_map.png}
    \caption{Distribución de las zonas de vida según la clasificación geográfica.}
    \label{fig:zonify-geografico}
\end{figure}

\begin{figure}[H]
    \centering
    \includegraphics[width=1.0\textwidth]{images/color_map_murcia.png}
    \caption{Distribución de las zonas de vida en la Región de Murcia según la clasificación geográfica.}
    \label{fig:zonify-murcia}
\end{figure}


\end{document}