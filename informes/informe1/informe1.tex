% Archivo: informe1.tex

\documentclass[11pt,a4paper]{article}

% Idioma y tipografías
\usepackage[spanish, es-noquoting, es-lcroman, es-tabla]{babel}
\usepackage[T1]{fontenc}
\usepackage[utf8]{inputenc}
\usepackage{lmodern}

% Maquetación y tipografía fina
\usepackage[a4paper,margin=2.5cm]{geometry}
\usepackage{microtype}
\usepackage{setspace}
\onehalfspacing

% Utilidades
\usepackage{csquotes}
\usepackage{graphicx}
\usepackage{xcolor}
\usepackage{booktabs}
\usepackage{siunitx}
\usepackage{amsmath,amssymb}
\usepackage{enumitem}
\setlist{nosep,leftmargin=*,labelsep=0.5em}

% Hipervínculos
\usepackage[hidelinks]{hyperref}
\hypersetup{
    pdftitle={Informe de Ontología Bioclimática},
    pdfauthor={Autor},
    pdfsubject={Ontología Bioclimática},
    pdfkeywords={ontologías, OWL, bioclimática, conocimiento, SPARQL}
}

% Bibliografía
\usepackage[
    backend=biber,
    style=ieee,
    sorting=nyt,
    maxbibnames=99
]{biblatex}
\addbibresource{bibliography.bib}

% Subfiguras
\usepackage{subcaption}

% Código (Turtle/SPARQL/OWL)
\usepackage{listings}
\lstdefinestyle{ontostyle}{
    basicstyle=\ttfamily\small,
    numbers=left,
    numberstyle=\tiny,
    frame=lines,
    breaklines=true,
    tabsize=2,
    columns=fullflexible,
    showstringspaces=false,
    keywordstyle=\color{blue!70!black}\bfseries,
    commentstyle=\color{gray!70},
    stringstyle=\color{green!50!black}
}
\lstdefinelanguage{Turtle}{
    morekeywords={@prefix,@base,a},
    sensitive=true,
    morecomment=[l]{\#},
    morestring=[b]{"}
}
\lstdefinelanguage{SPARQL}{
    morekeywords={SELECT,CONSTRUCT,ASK,DESCRIBE,WHERE,FROM,NAMED,PREFIX,BASE,OPTIONAL,FILTER,GRAPH,UNION,MINUS,BIND,VALUES,ORDER,BY,LIMIT,OFFSET,GROUP,HAVING,AS,SERVICE,UNDEF,DISTINCT,REDUCED},
    sensitive=true,
    morecomment=[l]{\#},
    morestring=[b]{"}
}
\lstdefinelanguage{RDFXML}{
    morestring=[b]",
    morecomment=[s]{<!--}{-->}
}
\lstset{style=ontostyle}

% Comandos útiles
\newcommand{\ontologyname}{Ontología Bioclimática}
\newcommand{\iri}[1]{\texttt{#1}}
\newcommand{\term}[1]{\textsf{#1}}

\begin{document}
\begin{titlepage}
    \centering
    % \includegraphics[height=2cm]{logo.png}\par\vspace{1cm} % Descomentar si hay logo
    {\Large Informe Técnico 1}\par\vspace{0.5cm}
    {\huge\bfseries \ontologyname\par}\vspace{0.5cm}
    \begin{tabular}{@{}ll@{}}
        Autor 1: & Juan Diego Gallego Nicolás \\
        Contacto 1: & jdiego.gallego@um.es \\
        Autor 2: & Óscar Vera López \\
        Contacto 2: & oscar.veral@um.es \\
        Profesor: & Rodrigo Martínez Béjar \\
        Asignatura: & Conocimiento y Razonamiento Aproximado \\
    \end{tabular}
    \vfill
    {\large \today}\par
\end{titlepage}

\pagenumbering{roman}
\begin{abstract}
Este informe presenta el diseño, implementación y evaluación de una ontología bioclímatica centrada entorno
al Sistema de Zonas de Vida de Holdridge. La ontología está orientada hacia la clasificación de zonas bioclimáticas
a partir de datos geográficos y climáticos de diferentes estaciones especializadas ubicadas en diversas localidades
del territorio español. Se detallan los requisitos, alcance y metodología seguida, así como las decisiones de diseño
conceptual y técnica.
\end{abstract}

\paragraph{Palabras clave:} ontologías; OWL; bioclimática; conocimiento; zonas de vida;
\clearpage

\tableofcontents
\clearpage
\pagenumbering{arabic}

\section{Introducción}
\color{blue!70!black} TODO
\color{black}

\section{Metodología}
Decidimos seguir la metodología de Ontology Development 101 \cite{NoyMcGuinness2001Ontology101} para la creación de nuestra ontología bioclimática. 
Esta metodología nos proporcionó una guía clara y estructurada para abordar el desarrollo de la ontología, 
desde la definición de los requisitos hasta la implementación y evaluación final con Protégé.

\subsection{Dominio y alcance}
El primer paso para la elaboración de la ontología es definir el dominio y alcance. En nuestro caso, el dominio es la bioclimatología, 
con un enfoque específico en el Sistema de Zonas de Vida de Holdridge. Un paso clave para determinar el alcance es plantear las preguntas 
de competencia que la ontología debe ser capaz de responder. Estas preguntas guían el diseño y aseguran que la ontología cubra los 
aspectos esenciales del dominio, permitiendo ser unos "tests" para su validación.

\subsubsection{Preguntas de competencia (CQs)}
Algunas de las preguntas que nuestra ontología debe ser capaz de responder son las siguientes:
\begin{enumerate}
    \item ¿Cómo influye la situación geográfica en el clima de una región?
    \item ¿Qué magnitudes climáticas son necesarias para clasificar una zona bioclimática según el sistema de Holdridge?
    \item ¿Alguna zona de la Región de Murcia se clasifica como "monte espiniso" según el sistema de Holdridge?
    \item ¿En qué rango de valores se puede encontrar la ABT (biotemperatura media anual)?
    \item Dada una ABT x y una APP z, ¿cuál es el PER (relación de evapotranspiración potencial) correspondiente?
    \item ¿Cuáles son los tipos de regiones latitudinales que existen?
    \item Si una estación climática en la región tropical está a una altitud de 4200 metros, ¿cuál sería su piso altitudinal correspondiente?
\end{enumerate}
Si bien la lista no es exhaustiva, estas preguntas de competencia cubren aspectos fundamentales del dominio bioclimático y guían el desarrollo de la ontología.

\subsection{Trabajos relacionados}

\subsubsection{WorldClim}

WorldClim \cite{FickHijmans2017WorldClim} \cite{WorldClimWebsite} es un conjunto de datos climáticos globales que proporciona información sobre variables climáticas a diferentes resoluciones espaciales. 
Estos datos son útiles para la validación y calibración de modelos bioclimáticos, así como para la identificación de patrones climáticos en diferentes regiones. 
A pesar de ser una base de datos y no una ontología per se, WorlClim define 19 variables bioclimáticas derivadas de datos mensuales de temperatura y precipitación, 
inspirando la utilizacizón de una taxonomía de variables climáticas para este proyecto.

\subsection{ENVO}

ENVO \cite{Buttigieg2013} es un volcabulario formal y estandarizado diseñado para describir de manera inequívoca el entorno de cualquier organismo o muestra biológica
Contiene miles de terminos/clases organizados jerarquicamente y representando biomas, características ambientales, y materiales presentes en 
la naturaleza. En ENVO, se definen biomas propios de la clasificación de de zonas de vida de Holdridge, lo que nos permite explorar las relaciones de estos
biomas con el resto del universo de conocimiento de las ciencias ambientales.


\subsection{Terminos relevantes}

A continuación, hemos realizado un analisis exhaustivo de los términos clave relacionados con el dominio bioclimático y el Sistema de Zonas de Vida de Holdridge.
Estos términos son fundamentales para la construcción de la ontología y su correcta representación del conocimiento en este ámbito. La lista incluye conceptos geográficos, climáticos y bioclimáticos esenciales para entender y modelar el dominio.

\noindent
\textbf{Términos relacionados con las magnitudes geográficas:}

\begin{itemize}
    \item Superficie Terrestre
    \item Coordenadas Geográficas.
    \item Latitud.
    \item Unidad de medida de la latitud (grados).
    \item Paralelos.
    \item Línea del Ecuador.
    \item Trópico de Cáncer.
    \item Trópico de Capricornio.
    \item Polos terrestres.
    \item Círculo Polar Ártico.
    \item Círculo Polar Antártico.
    \item Longitud.
    \item Unidad de medida de la longitud (grados).
    \item Meridiano.
    \item Meridiano de Greenwich.
    \item Altitud.
    \item Unidad de Medida de Altitud (metros).
    \item Nivel del mar.
    
\end{itemize}

\noindent
\textbf{Términos relacionados con las magnitudes climáticas y su medición:}

\begin{itemize}
    \item Estación meteorológica o climática.
    \item Variable climática o atmosférica.
    \item Temperatura.
    \item Precipitación.
    \item Viento (dirección y fuerza).
    \item Presión.
    \item Humedad.
    \item Radiación solar.
    \item Sofisticación de la estación climática (mucho o poco).
    \item Equipamiento de la estación bioclimática.
    \item Sensores climáticos.
    \item Termómetro.
    \item Higrómetro.
    \item Pluviómetro.
    \item Anemómetro.
    \item Barómetro.
    \item Piranómetro.
    \item Sensor UV.
    \item Registros de estación climática.
    \item Registro de valor medio anual.
    \item Registro de valor medio mensual.
    \item Tipo de variable climática.
    \item Variable climatica de medición directa.
    \item Variable climática derivada.
    \item Unidades de medida de los sensores y variables climáticas.
\end{itemize}

\noindent
\textbf{Términos relacionados con la bioclimatología y el Sistema de Zonas de Vida de Holdridge:}

\begin{itemize}
    \item Franja clímatica.
    \item Zona cálida o trópical.
    \item Zona templada.
    \item Zona fría.
    \item Clima.
    \item Vegetación.
    \item Fauna.
    \item Territorio.
    \item Ecosistema.
    \item Clasificación bioclimática.
    \item Factores ecológicos.
    \item Zona ecológica.
    \item Zonas de vida.
    \item Parámetros climáticos.
    \item Tipo de suelo.
    \item Acción humana.
    \item Sistema de clasificación bioclimática.
    \item Sistema de Zonas de Vida de Holdridge.
    \item Variable climática fundamental.
    \item Biotemperatura media anual (ABT).
    \item Promedio de Precipitación anual (APP).
    \item Relación de Evapotranspiración Potencial (PER).
    \item Región Latitudinal.
    \item Provincias de Humedad.
    \item Pisos altitudinales.
    \item Restricciones sobre los parámetros climáticos en cada zona.
    \item Biotemperatura.
    \item Calor efectivo.
    \item Rangos de Biotemperatura.
    \item Piso Basal.
    \item Tipo de precipitación (agua, nieve o granizo).
    \item Formula de derivación de APP.
    \item Evapotranspiración.
    \item Evapotranspiración Potencial (APE).
    \item Formula de derivación de APE.
    \item Formula de derivación de PER.

\end{itemize}

\subsubsection{BabelNet}
BabelNet \cite{navigli-ponzetto-2010-babelnet} es una red semántica multilingüe que integra información léxica y enciclopédica de diversas fuentes.
Aunque BabelNet no es una ontología específica del dominio bioclimático, su amplia cobertura de términos y conceptos puede ser útil para enriquecer la ontología, 
además de ser su uso obligatorio para la elaboración de la práctica. 

Como primera aproximación, utilizamos su buscador para introducir uno de los términos de nuestro dominio: "Life Zone" \ref{fig:babelnet-life-zone}. 
El término se muestra como el centro de un grafo radial cuyos vecinos se encuentran conectados mediante aristas de mayor o menor longitud dependiendo
de un criterio de similitud semántica. En este caso, se observa que "Life Zone" está relacionado con conceptos como "vegetation zone" o, 
de manera íntima con "Holdridge life zone", que es precisamente el núcleo de nuestro dominio.
\begin{figure}[h]
    \centering
    \includegraphics[width=0.8\linewidth]{images/bbn_lifezone.png}
    \caption{Resultado de la búsqueda del término ''Life Zone'' en BabelNet.}
    \label{fig:babelnet-life-zone}
\end{figure}

Haciendo clic en el nodo, se muestran los conceptos relacionados con la teoría de las Zonas de Vida de Holdridge \ref{fig:babelnet-holdridge}, como pueden ser
"altitudinal zonation", "vaporization", "humidity", "polar desert"... 

\begin{figure}[h]
    \centering
    \includegraphics[width=0.8\linewidth]{images/bbn_holdridge.png}
    \caption{Conceptos relacionados con ''Holdridge life zone'' en BabelNet.}
    \label{fig:babelnet-holdridge}
\end{figure}

\subsection{Definición de las clases (conceptos) y su jerarquía.}
Para el diseño de la jerarquía conceptual, vamos a seguir un enfoque top-down,
partiendo de conceptos generales hacia conceptos más específicos. 
A continuación, se presentan las clases principales y su jerarquía:

\subsubsection{Taxonomía: ZonaVidaHoldridge}
El concepto clave de nuestra ontología es la Zona de Vida de Holdridge, una clasificación bioclimática
que categoriza las regiones del mundo según su clima y posición geográfica. 
A su vez, ZonaVidaHoldridge es una taxonomía que agrupa las diferentes zonas bioclimáticas definidas por Holdridge tales como: \\ \\
\begin{tabular}{@{}llll@{}}
    \toprule
    \term{Desierto} & \term{TundraSeca} & \term{TundraHumeda} & \term{TundraPluvial} \\
    \term{BosquePluvial} & \term{BosqueHumedo} & \term{MatorralDesertico} & \term{Estepa} \\
    \term{BosqueMuyHumedo} & \term{BosqueSeco} & \term{MonteEspinoso} & \term{} \\
    \bottomrule
\end{tabular} \\ \\
Cada una de estas clases viene determinada por tres magnitudes climáticas fundamentales:
\begin{itemize}
    \item Biotemperatura Media Anual (ABT).
    \item Precipitación Media Anual (APP).
    \item Relación de Evapotranspiración Potencial (PER).
\end{itemize}
Estas magnitudes definen los límites y características de cada zona bioclimática y están estrechamente relacionadas con la ubicación geográfica y las condiciones climáticas de la región.

\subsubsection{Taxonomía: MagnitudBioclimatica}
Este concepto abstracto engloba las diferentes magnitudes bioclimáticas y que se utilizan para definir y clasificar las zonas de vida de Holdridge.
Encontramos:

\begin{itemize}
    \item Biotemperatura Media Anual (ABT).
    \item Precipitación Media Anual (APP).
    \item Relación de Evapotranspiración Potencial (PER).
\end{itemize}

También consideramos, por su relación con las anteriores, la magnitud de Evapotranspiración Potencial (APE).
Todas estas magnitudes se instanciarán como variables difusas en nuestra ontología, permitiendo representar la incertidumbre inherente a los datos climáticos y geográficos.

No incluimos aquí magnitudes climáticas directas como Temperatura o Precipitación, ya que estas se modelarán como atributos de las estaciones climáticas.

\subsubsection{Taxonomía/Topología: RegionLatitudinal}
Este concepto agrupa las diferentes regiones latitudinales que influyen en la clasificación bioclimática según el sistema de Holdridge. 
Las regiones latitudinales son categorías geográficas divididas según paralelos específicos. Estas regiones mantinenen una relación 
topológica entre sí, ya que están organizadas de norte a sur en función de su latitud.
Las regiones latitudinales consideradas son:
\begin{itemize}
    \item Tropical
    \item Subtropical
    \item TempladaCaliente
    \item TempladaFria
    \item Boreal
    \item Subpolar
    \item Polar
\end{itemize}

\subsubsection{PisoAltitudinal}
Piso altitudinal es el otro concepto geográfico clave en la clasificación bioclimática de Holdridge.
Los pisos altitudinales son zonas definidas por rangos trapezoidales de altitud y longitud y que Holdridge asoció
a diferentes rangos de ABT. Los pisos altitudinales considerados son:
\begin{itemize}
    \item Basal
    \item Premontano
    \item Montano 
    \item Subalpino
    \item Alpino
    \item Nival
\end{itemize}

\subsection{Atributos}
Cada clase definida anteriormente tendrá atributos específicos que describen sus características y propiedades.




















\subsection{Casos de uso}
Describa escenarios de uso, fuentes de datos y actores.

\section{Metodología}
Describa el proceso (p.~ej., METHONTOLOGY/NeOn): especificación de requisitos, conceptualización, formalización, implementación, evaluación, mantenimiento y publicación.

\section{Diseño conceptual}
\subsection{Alcance y límites}
Defina contexto, granularidad espacial/temporal y supuestos.

\subsection{Clases principales}
Ejemplos: \term{ZonaBioclimática}, \term{VariableClimática}, \term{Indicador}, \term{Observación}, \term{Estación}, \term{Región}, \term{Taxón}.

\subsection{Propiedades y restricciones}
Cardinalidades, dominios/rangos, axiomas clave (disjunción, equivalencias).

\section{Reuso y alineación}
Alinee con vocabularios externos (p.~ej., \textit{SOSA/SSN}, \textit{GeoSPARQL}). Documente mapeos y decisiones.

\section{Implementación en OWL}
\subsection{Prefijos y espacios de nombres}
\noindent
\begin{lstlisting}[language=Turtle,caption={Prefijos base en Turtle.}]
@prefix :      <https://ejemplo.org/bioclima#> .
@prefix owl:   <http://www.w3.org/2002/07/owl#> .
@prefix rdf:   <http://www.w3.org/1999/02/22-rdf-syntax-ns#> .
@prefix rdfs:  <http://www.w3.org/2000/01/rdf-schema#> .
@prefix xsd:   <http://www.w3.org/2001/XMLSchema#> .
@prefix sosa:  <http://www.w3.org/ns/sosa/> .
@prefix geo:   <http://www.opengis.net/ont/geosparql#> .
\end{lstlisting}

\subsection{Clases y propiedades}
\begin{lstlisting}[language=Turtle,caption={Fragmento de clases y propiedades.}]
:ZonaBioclimatica a owl:Class ;
    rdfs:label "Zona bioclimatica"@es .

:VariableClimatica a owl:Class ;
    rdfs:label "Variable climatica"@es .

:Indicador a owl:Class ;
    rdfs:label "Indicador bioclimatico"@es .

:observaVariable a owl:ObjectProperty ;
    rdfs:domain sosa:Observation ;
    rdfs:range  :VariableClimatica ;
    rdfs:label "observa variable"@es .

:valorIndicador a owl:DatatypeProperty ;
    rdfs:domain :Indicador ;
    rdfs:range  xsd:decimal ;
    rdfs:label "valor de indicador"@es .
\end{lstlisting}

\subsection{Individuos de ejemplo}
\begin{lstlisting}[language=Turtle,caption={Individuos ilustrativos.}]
:Aridez a :Indicador ;
    rdfs:label "Indice de Aridez"@es .

:ZB_Mediterranea a :ZonaBioclimatica ;
    rdfs:label "Zona Bioclimatica Mediterranea"@es .
\end{lstlisting}

\section{Consultas SPARQL}
\begin{lstlisting}[language=SPARQL,caption={Zonas bioclimáticas por región.}]
PREFIX :    <https://ejemplo.org/bioclima#>
PREFIX rdfs:<http://www.w3.org/2000/01/rdf-schema#>

SELECT ?zona ?nombre WHERE {
    ?zona a :ZonaBioclimatica ;
                rdfs:label ?nombre .
    # Agregue filtros geoespaciales si aplica (GeoSPARQL)
}
ORDER BY ?nombre
\end{lstlisting}

\begin{lstlisting}[language=SPARQL,caption={Indicadores y valores promedio.}]
PREFIX :   <https://ejemplo.org/bioclima#>
PREFIX xsd:<http://www.w3.org/2001/XMLSchema#>

SELECT ?indicador (AVG(?valor) AS ?promedio) WHERE {
    ?i a :Indicador ;
         rdfs:label ?indicador ;
         :valorIndicador ?valor .
}
GROUP BY ?indicador
\end{lstlisting}

\section{Evaluación}
Verificación (consistencia, coherencia, competencia) y validación (con expertos). Use reasoners (HermiT, Pellet) y pruebas contra CQs.

\section{Publicación y acceso}
Estrategia FAIR: documentación, serializaciones (TTL, RDF/XML), resolución de URIs, endpoint SPARQL, metadatos VoID/DCAT, versionado.

\section{Resultados y discusión}
Análisis de cobertura, utilidad en casos de uso, limitaciones y amenazas a la validez.

\section{Conclusiones y trabajo futuro}
Síntesis de aportes, líneas de mejora, mantenimiento y extensiones planificadas.

\section*{Agradecimientos}
Reconocimientos a colaboradores, instituciones y fuentes de datos.

\appendix
\section{Anexo A: Figuras y tablas}
% Ejemplo de figura (descomente e incluya un recurso válido)
% \begin{figure}[h]
%   \centering
%   \includegraphics[width=0.8\linewidth]{figuras/mapa_zonas.png}
%   \caption{Mapa de zonas bioclimáticas.}
%   \label{fig:mapa-zonas}
% \end{figure}

\appendix
\section{Anexo B: Variables climáticas}
Tabla con variables climáticas modeladas en la ontología. Se indica nombre, abreviatura y unidad.
\begin{table}[h]
    \centering
    \caption{Variables climáticas modeladas.}
    \begin{tabular}{@{}lll@{}}
        \toprule
        Variable & Abreviatura & Unidad \\
        \midrule
        Latitud & & \si{\degree} \\
        Longitud & & \si{\degree} \\
        Altitud (sobre el nivel del mar) & & \si{\meter} \\
        Biotemperatura media anual & ABT & \si{\degreeCelsius} \\
        Precipitación media anual & APP & \si{\milli\metre} \\
        Relación de evapotranspiración potencial & PER & \textit{adimensional} \\
        Evapotranspiración potencial & APE & \si{\milli\metre} \\
        \bottomrule
    \end{tabular}
\end{table}

\clearpage
\printbibliography

\end{document}