% Archivo: informe1.tex

\documentclass[11pt,a4paper]{article}

% Idioma y tipografías
\usepackage[spanish, es-noquoting, es-lcroman, es-tabla]{babel}
\usepackage[T1]{fontenc}
\usepackage[utf8]{inputenc}
\usepackage{lmodern}

% Maquetación y tipografía fina
\usepackage[a4paper,margin=2.5cm]{geometry}
\usepackage{microtype}
\usepackage{setspace}
\onehalfspacing

% Utilidades
\usepackage{csquotes}
\usepackage{graphicx}
\usepackage{xcolor}
\usepackage{booktabs}
\usepackage{siunitx}
\usepackage{amsmath,amssymb}
\usepackage{enumitem}
\setlist{nosep,leftmargin=*,labelsep=0.5em}

% Hipervínculos
\usepackage[hidelinks]{hyperref}
\hypersetup{
    pdftitle={Informe de Ontología Bioclimática},
    pdfauthor={Autor},
    pdfsubject={Ontología Bioclimática},
    pdfkeywords={ontologías, OWL, bioclimática, conocimiento, SPARQL}
}

% Bibliografía
\usepackage[
    backend=biber,
    style=ieee,
    sorting=nyt,
    maxbibnames=99
]{biblatex}
\addbibresource{bibliography.bib}

% Subfiguras
\usepackage{subcaption}

% Código (Turtle/SPARQL/OWL)
\usepackage{listings}
\lstdefinestyle{ontostyle}{
    basicstyle=\ttfamily\small,
    numbers=left,
    numberstyle=\tiny,
    frame=lines,
    breaklines=true,
    tabsize=2,
    columns=fullflexible,
    showstringspaces=false,
    keywordstyle=\color{blue!70!black}\bfseries,
    commentstyle=\color{gray!70},
    stringstyle=\color{green!50!black}
}
\lstdefinelanguage{Turtle}{
    morekeywords={@prefix,@base,a},
    sensitive=true,
    morecomment=[l]{\#},
    morestring=[b]{"}
}
\lstdefinelanguage{SPARQL}{
    morekeywords={SELECT,CONSTRUCT,ASK,DESCRIBE,WHERE,FROM,NAMED,PREFIX,BASE,OPTIONAL,FILTER,GRAPH,UNION,MINUS,BIND,VALUES,ORDER,BY,LIMIT,OFFSET,GROUP,HAVING,AS,SERVICE,UNDEF,DISTINCT,REDUCED},
    sensitive=true,
    morecomment=[l]{\#},
    morestring=[b]{"}
}
\lstdefinelanguage{RDFXML}{
    morestring=[b]",
    morecomment=[s]{<!--}{-->}
}
\lstset{style=ontostyle}

% Comandos útiles
\newcommand{\ontologyname}{Ontología Bioclimática}
\newcommand{\iri}[1]{\texttt{#1}}
\newcommand{\term}[1]{\textsf{#1}}

\begin{document}
\begin{titlepage}
    \centering
    % \includegraphics[height=2cm]{logo.png}\par\vspace{1cm} % Descomentar si hay logo
    {\Large Informe Técnico 1}\par\vspace{0.5cm}
    {\huge\bfseries \ontologyname\par}\vspace{0.5cm}
    \begin{tabular}{@{}ll@{}}
        Autor 1: & Juan Diego Gallego Nicolás \\
        Contacto 1: & jdiego.gallego@um.es \\
        Autor 2: & Óscar Vera López \\
        Contacto 2: & oscar.veral@um.es \\
        Profesor: & Rodrigo Martínez Béjar \\
        Asignatura: & Conocimiento y Razonamiento Aproximado \\
    \end{tabular}
    \vfill
    {\large \today}\par
\end{titlepage}

\pagenumbering{roman}
\begin{abstract}
Este informe presenta el diseño, implementación y evaluación de una ontología bioclímatica centrada entorno
al Sistema de Zonas de Vida de Holdridge. La ontología está orientada hacia la clasificación de zonas bioclimáticas
a partir de datos geográficos y climáticos de diferentes estaciones especializadas ubicadas en diversas localidades
del territorio español. Se detallan los requisitos, alcance y metodología seguida, así como las decisiones de diseño
conceptual y técnica.
\end{abstract}

\paragraph{Palabras clave:} ontologías; OWL; bioclimática; conocimiento; zonas de vida;
\clearpage

\tableofcontents
\clearpage
\pagenumbering{arabic}

\section{Introducción}
\color{blue!70!black} TODO
\color{black}

\section{Metodología}
Decidimos seguir la metodología de Ontology Development 101 \cite{NoyMcGuinness2001Ontology101} para la creación de nuestra ontología bioclimática. 
Esta metodología nos proporcionó una guía clara y estructurada para abordar el desarrollo de la ontología, 
desde la definición de los requisitos hasta la implementación y evaluación final con Protégé.

\subsection{Dominio y alcance}
El primer paso para la elaboración de la ontología es definir el dominio y alcance. En nuestro caso, el dominio es la bioclimatología, 
con un enfoque específico en el Sistema de Zonas de Vida de Holdridge. Un paso clave para determinar el alcance es plantear las preguntas 
de competencia que la ontología debe ser capaz de responder. Estas preguntas guían el diseño y aseguran que la ontología cubra los 
aspectos esenciales del dominio, permitiendo ser unos "tests" para su validación.

\subsubsection{Preguntas de competencia (CQs)}
Algunas de las preguntas que nuestra ontología debe ser capaz de responder son las siguientes:
\begin{enumerate}
    \item ¿Cómo influye la situación geográfica en el clima de una región?
    \item ¿Qué magnitudes climáticas son necesarias para clasificar una zona bioclimática según el sistema de Holdridge?
    \item ¿Alguna zona de la Región de Murcia se clasifica como "monte espiniso" según el sistema de Holdridge?
    \item ¿En qué rango de valores se puede encontrar la ABT (biotemperatura media anual)?
    \item Dada una ABT x y una APP z, ¿cuál es el PER (relación de evapotranspiración potencial) correspondiente?
    \item ¿Cuáles son los tipos de regiones latitudinales que existen?
    \item Si una estación climática en la región tropical está a una altitud de 4200 metros, ¿cuál sería su piso altitudinal correspondiente?
\end{enumerate}
Si bien la lista no es exhaustiva, estas preguntas de competencia cubren aspectos fundamentales del dominio bioclimático y guían el desarrollo de la ontología.

\subsection{Trabajos relacionados}

\subsubsection{WorldClim}

WorldClim \cite{FickHijmans2017WorldClim} \cite{WorldClimWebsite} es un conjunto de datos climáticos globales que proporciona información sobre variables climáticas a diferentes resoluciones espaciales. 
Estos datos son útiles para la validación y calibración de modelos bioclimáticos, así como para la identificación de patrones climáticos en diferentes regiones. 
A pesar de ser una base de datos y no una ontología per se, WorlClim define 19 variables bioclimáticas derivadas de datos mensuales de temperatura y precipitación, 
inspirando la utilizacizón de una taxonomía de variables climáticas para este proyecto.

\subsection{ENVO}

ENVO \cite{Buttigieg2013} es un volcabulario formal y estandarizado diseñado para describir de manera inequívoca el entorno de cualquier organismo o muestra biológica
Contiene miles de terminos/clases organizados jerarquicamente y representando biomas, características ambientales, y materiales presentes en 
la naturaleza. En ENVO, se definen biomas propios de la clasificación de de zonas de vida de Holdridge, lo que nos permite explorar las relaciones de estos
biomas con el resto del universo de conocimiento de las ciencias ambientales.

\section{Terminos relevantes}

\begin{itemize}
    \item Latitud
    \item Superficie Terrestre
    \item Linea del ecuador
    \item Unidad de medida de la latitud
    \item Paralelos
    \item Franja clímatica
    \item Zona cálida o trópical
    \item Trópico de Cancer
    \item Trópico de Capricornio
    \item Zona templada
    \item Circulo Polar
    \item Zona fria
    \item Polos terrestres
    \item Circulo Polar Ártico
    \item Circulo Polar Antártico
    \item Estación Meteorológica o Climática
    \item Variables climáticas o atmosféricas
    \item Temperatura
    \item Precipitación
    \item Viento (Dirección y Fuerza)
    \item Presión
    \item Humedad
    \item Radiación solar
    \item Coordenadas Geográficas
    \item Longitud
    \item Altitud
    \item Meridiano de Greenwich
    \item Unidad de Medida de Longitud
    \item Direcciónes Cardinalidades (Norte, Sur, Este, Oeste)
    \item Meridiano
    \item Unidad de Medida de Altitud
    \item Nivel del Mar
    \item Sofisticación de la Estación Climática (mucho, poco)
    \item Equipamiento de la Estación Bioclimática
    \item Objetivo de la Estación bioclímatica
    \item Sensores Climáticos
    \item Termómetro
    \item Higrómetro
    \item Pluviómetro
    \item Anemómetro
    \item Barómetro
    \item Piranómetro
    \item Sensor UV
    \item Unidades de Medida de los sensores y variables climáticas.
    \item PONER TODAS LAS ESTACIONES DE ESPAÑA COMO INSTANCIAS (AÑADIR IDENTIFICADORES COMO PROPIEDAD)
    \item Registros de estación climática
    \item Registro de valor medio anual
    \item Registro de valor medio mensual
    \item Tipo de variable climática
    \item Variable climatica de medición directa
    \item Variable climática derivada
    \item Clima
    \item Vegetación
    \item Fauna
    \item territorio
    \item Ecosistema
    \item Clasificación Bioclimática
    \item Factores ecoloógicos
    \item Mapa
    \item Modelo de elevación digital
    \item Zona Ecológica
    \item Sistema de Clasificación Bioclimatica
    \item Sistema de Zonas de Vida de Holdridge
    \item Variable Climática fundamental
    \item (Como relacíon, se requiren de registros de estaciones climaticas para su aplicación)
    \item Zonas de Vida
    \item Parámetros climáticos
    \item Tipo de suelo
    \item Acción humana
    \item (Relación explícita entre vegetacióon y clima)
    \item Biotemperatura media anual (ABT)
    \item Promedio de Precipitación anual (APP)
    \item Relación de Evapotranspiración Potencial (PER) (relacionado humedad ambiental)
    \item Región Latitudinal (POENR TODA LA LISTA)
    \item Provincias de Humedad (PONER TODA LA LISTA)
    \item Pisos altitudinales (PONER TODA LA LISTA) (Alturas de caracter local)
    \item LISTA DE ZONAS DE VIDA (PONER TODA LA LISTA)
    \item Restricciones sobre los parámetros climáticos en cada zona
    \item Biotemperatura
    \item Calor efectivo
    \item Rangos de Biotemperatura
    \item Relación Latitud-Temperatura-ABT -> Regiones latitudinales
    \item Relación Altitud-Temperatura-ABT -> Pisos altitudinales
    \item Piso Basal
    \item Relación Piso Basal - Region Latitudinal - Cantidad de Pisos Altitudinales (y zonas de vida) posibles
    \item Tipo de precipitación (Agua, Nieve o Granizo)
    \item Formula cálculo APP
    \item Evapotranspiración
    \item Evapotranspiración Potencial (APE) (relación humedad ambiental)
    \item Formulas relación APE = 58,93 x ABT y PER = APE/APP


\end{itemize}

\subsection{Definición de las clases (conceptos) y su jerarquía.}
Para el diseño de la jerarquía conceptual, vamos a seguir un enfoque top-down,
partiendo de conceptos generales hacia conceptos más específicos. 
A continuación, se presentan las clases principales y su jerarquía:

\subsubsection{Taxonomía: Zona de Vida de Holdridge}
El concepto clave de nuestra ontología es la Zona de Vida de Holdridge, una clasificación bioclimática
que categoriza las regiones del mundo según su clima y posición geográfica. No obstante,
como el una ontología debe aspirar a ser reutilizable y extensible, hemos decidido no establecer directamente
este concepto como raíz, sino que vamos a definir una jerarquía más amplia que permita incorporar otros conceptos relacionados.
Así pues, proponemos el concepto raíz ZonaBioclimática, que engloba todas las zonas bioclimáticas, incluyendo las de Holdridge.
Como especialización, definimos la clase ZonaVidaHoldridge como subclase de ZonaBioclimática.
A su vez, ZonaVidaHoldridge se divide en los diferentes tipos de zonas de vida, tales como:
\begin{tabular}{@{}llll@{}}
    \toprule
    \term{Desierto} & \term{TundraSeca} & \term{TundraHumeda} & \term{TundraPluvial} \\
    \term{BosquePluvial} & \term{BosqueHumedo} & \term{MatorralDesertico} & \term{Estepa} \\
    \bottomrule
\end{tabular}



















\subsection{Casos de uso}
Describa escenarios de uso, fuentes de datos y actores.

\section{Metodología}
Describa el proceso (p.~ej., METHONTOLOGY/NeOn): especificación de requisitos, conceptualización, formalización, implementación, evaluación, mantenimiento y publicación.

\section{Diseño conceptual}
\subsection{Alcance y límites}
Defina contexto, granularidad espacial/temporal y supuestos.

\subsection{Clases principales}
Ejemplos: \term{ZonaBioclimática}, \term{VariableClimática}, \term{Indicador}, \term{Observación}, \term{Estación}, \term{Región}, \term{Taxón}.

\subsection{Propiedades y restricciones}
Cardinalidades, dominios/rangos, axiomas clave (disjunción, equivalencias).

\section{Reuso y alineación}
Alinee con vocabularios externos (p.~ej., \textit{SOSA/SSN}, \textit{GeoSPARQL}). Documente mapeos y decisiones.

\section{Implementación en OWL}
\subsection{Prefijos y espacios de nombres}
\noindent
\begin{lstlisting}[language=Turtle,caption={Prefijos base en Turtle.}]
@prefix :      <https://ejemplo.org/bioclima#> .
@prefix owl:   <http://www.w3.org/2002/07/owl#> .
@prefix rdf:   <http://www.w3.org/1999/02/22-rdf-syntax-ns#> .
@prefix rdfs:  <http://www.w3.org/2000/01/rdf-schema#> .
@prefix xsd:   <http://www.w3.org/2001/XMLSchema#> .
@prefix sosa:  <http://www.w3.org/ns/sosa/> .
@prefix geo:   <http://www.opengis.net/ont/geosparql#> .
\end{lstlisting}

\subsection{Clases y propiedades}
\begin{lstlisting}[language=Turtle,caption={Fragmento de clases y propiedades.}]
:ZonaBioclimatica a owl:Class ;
    rdfs:label "Zona bioclimatica"@es .

:VariableClimatica a owl:Class ;
    rdfs:label "Variable climatica"@es .

:Indicador a owl:Class ;
    rdfs:label "Indicador bioclimatico"@es .

:observaVariable a owl:ObjectProperty ;
    rdfs:domain sosa:Observation ;
    rdfs:range  :VariableClimatica ;
    rdfs:label "observa variable"@es .

:valorIndicador a owl:DatatypeProperty ;
    rdfs:domain :Indicador ;
    rdfs:range  xsd:decimal ;
    rdfs:label "valor de indicador"@es .
\end{lstlisting}

\subsection{Individuos de ejemplo}
\begin{lstlisting}[language=Turtle,caption={Individuos ilustrativos.}]
:Aridez a :Indicador ;
    rdfs:label "Indice de Aridez"@es .

:ZB_Mediterranea a :ZonaBioclimatica ;
    rdfs:label "Zona Bioclimatica Mediterranea"@es .
\end{lstlisting}

\section{Consultas SPARQL}
\begin{lstlisting}[language=SPARQL,caption={Zonas bioclimáticas por región.}]
PREFIX :    <https://ejemplo.org/bioclima#>
PREFIX rdfs:<http://www.w3.org/2000/01/rdf-schema#>

SELECT ?zona ?nombre WHERE {
    ?zona a :ZonaBioclimatica ;
                rdfs:label ?nombre .
    # Agregue filtros geoespaciales si aplica (GeoSPARQL)
}
ORDER BY ?nombre
\end{lstlisting}

\begin{lstlisting}[language=SPARQL,caption={Indicadores y valores promedio.}]
PREFIX :   <https://ejemplo.org/bioclima#>
PREFIX xsd:<http://www.w3.org/2001/XMLSchema#>

SELECT ?indicador (AVG(?valor) AS ?promedio) WHERE {
    ?i a :Indicador ;
         rdfs:label ?indicador ;
         :valorIndicador ?valor .
}
GROUP BY ?indicador
\end{lstlisting}

\section{Evaluación}
Verificación (consistencia, coherencia, competencia) y validación (con expertos). Use reasoners (HermiT, Pellet) y pruebas contra CQs.

\section{Publicación y acceso}
Estrategia FAIR: documentación, serializaciones (TTL, RDF/XML), resolución de URIs, endpoint SPARQL, metadatos VoID/DCAT, versionado.

\section{Resultados y discusión}
Análisis de cobertura, utilidad en casos de uso, limitaciones y amenazas a la validez.

\section{Conclusiones y trabajo futuro}
Síntesis de aportes, líneas de mejora, mantenimiento y extensiones planificadas.

\section*{Agradecimientos}
Reconocimientos a colaboradores, instituciones y fuentes de datos.

\appendix
\section{Anexo A: Figuras y tablas}
% Ejemplo de figura (descomente e incluya un recurso válido)
% \begin{figure}[h]
%   \centering
%   \includegraphics[width=0.8\linewidth]{figuras/mapa_zonas.png}
%   \caption{Mapa de zonas bioclimáticas.}
%   \label{fig:mapa-zonas}
% \end{figure}

\appendix
\section{Anexo B: Variables climáticas}
Tabla con variables climáticas modeladas en la ontología. Se indica nombre, abreviatura y unidad.
\begin{table}[h]
    \centering
    \caption{Variables climáticas modeladas.}
    \begin{tabular}{@{}lll@{}}
        \toprule
        Variable & Abreviatura & Unidad \\
        \midrule
        Latitud & & \si{\degree} \\
        Longitud & & \si{\degree} \\
        Altitud (sobre el nivel del mar) & & \si{\meter} \\
        Biotemperatura media anual & ABT & \si{\degreeCelsius} \\
        Precipitación media anual & APP & \si{\milli\metre} \\
        Relación de evapotranspiración potencial & PER & \textit{adimensional} \\
        Evapotranspiración potencial & APE & \si{\milli\metre} \\
        \bottomrule
    \end{tabular}
\end{table}

\clearpage
\printbibliography

\end{document}