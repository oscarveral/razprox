% Archivo: informe1.tex

\documentclass[11pt,a4paper]{article}

% Idioma y tipografías
\usepackage[spanish, es-noquoting, es-lcroman, es-tabla]{babel}
\usepackage[T1]{fontenc}
\usepackage[utf8]{inputenc}
\usepackage{lmodern}

% Maquetación y tipografía fina
\usepackage[a4paper,margin=2.5cm]{geometry}
\usepackage{microtype}
\usepackage{setspace}
\onehalfspacing

% Utilidades
\usepackage{csquotes}
\usepackage{graphicx}
\usepackage{xcolor}
\usepackage{booktabs}
\usepackage{siunitx}
\usepackage{amsmath,amssymb}
\usepackage{enumitem}
\setlist{nosep,leftmargin=*,labelsep=0.5em}

% Hipervínculos
\usepackage[hidelinks]{hyperref}
\hypersetup{
    pdftitle={Informe de Ontología Bioclimática},
    pdfauthor={Autor},
    pdfsubject={Ontología Bioclimática},
    pdfkeywords={ontologías, OWL, bioclimática, conocimiento, SPARQL}
}

% Bibliografía
\usepackage[
    backend=biber,
    style=ieee,
    sorting=nyt,
    maxbibnames=99
]{biblatex}
\addbibresource{bibliography.bib}

% Subfiguras
\usepackage{subcaption}

% Código (Turtle/SPARQL/OWL)
\usepackage{listings}
\lstdefinestyle{ontostyle}{
    basicstyle=\ttfamily\small,
    numbers=left,
    numberstyle=\tiny,
    frame=lines,
    breaklines=true,
    tabsize=2,
    columns=fullflexible,
    showstringspaces=false,
    keywordstyle=\color{blue!70!black}\bfseries,
    commentstyle=\color{gray!70},
    stringstyle=\color{green!50!black}
}
\lstdefinelanguage{Turtle}{
    morekeywords={@prefix,@base,a},
    sensitive=true,
    morecomment=[l]{\#},
    morestring=[b]{"}
}
\lstdefinelanguage{SPARQL}{
    morekeywords={SELECT,CONSTRUCT,ASK,DESCRIBE,WHERE,FROM,NAMED,PREFIX,BASE,OPTIONAL,FILTER,GRAPH,UNION,MINUS,BIND,VALUES,ORDER,BY,LIMIT,OFFSET,GROUP,HAVING,AS,SERVICE,UNDEF,DISTINCT,REDUCED},
    sensitive=true,
    morecomment=[l]{\#},
    morestring=[b]{"}
}
\lstdefinelanguage{RDFXML}{
    morestring=[b]",
    morecomment=[s]{<!--}{-->}
}
\lstset{style=ontostyle}

% Comandos útiles
\newcommand{\ontologyname}{Ontología Bioclimática}
\newcommand{\iri}[1]{\texttt{#1}}
\newcommand{\term}[1]{\textsf{#1}}

\begin{document}
\begin{titlepage}
    \centering
    % \includegraphics[height=2cm]{logo.png}\par\vspace{1cm} % Descomentar si hay logo
    {\Large Informe Técnico 1}\par\vspace{0.5cm}
    {\huge\bfseries \ontologyname\par}\vspace{0.5cm}
    \begin{tabular}{@{}ll@{}}
        Autor 1: & Juan Diego Gallego Nicolás \\
        Contacto 1: & jdiego.gallego@um.es \\
        Autor 2: & Óscar Vera López \\
        Contacto 2: & oscar.veral@um.es \\
        Profesor: & Rodrigo Martínez Béjar \\
        Asignatura: & Conocimiento y Razonamiento Aproximado \\
    \end{tabular}
    \vfill
    {\large \today}\par
\end{titlepage}

\pagenumbering{roman}
\begin{abstract}
Este informe presenta el diseño, implementación y evaluación de una ontología bioclímatica centrada entorno
al Sistema de Zonas de Vida de Holdridge. La ontología está orientada hacia la clasificación de zonas bioclimáticas
a partir de datos geográficos y climáticos de diferentes estaciones especializadas ubicadas en diversas localidades
del territorio español. Se detallan los requisitos, alcance y metodología seguida, así como las decisiones de diseño
conceptual y técnica.
\end{abstract}

\paragraph{Palabras clave:} ontologías; OWL; bioclimática; conocimiento; zonas de vida;
\clearpage

\tableofcontents
\clearpage
\pagenumbering{arabic}

\section{Introducción}
\color{blue!70!black} TODO
\color{black}

\section{Metodología}
Decidimos seguir la metodología de Ontology Development 101 \cite{NoyMcGuinness2001Ontology101} para la creación de nuestra ontología bioclimática. 
Esta metodología nos proporcionó una guía clara y estructurada para abordar el desarrollo de la ontología, 
desde la definición de los requisitos hasta la implementación y evaluación final.

\subsection{Dominio y alcance}
\subsubsection{Competency Questions (CQs)}
Liste preguntas que la ontología debe responder:
\begin{enumerate}
    \item ¿Cómo influye la situación geográfica en el clima de una región?
    \item ¿Qué magnitudes climáticas son necesarias para clasificar una zona bioclimática según el sistema de Holdridge?
    \item ¿Alguna zona de la Región de Murcia se clasifica como "monte espiniso" según el sistema de Holdridge?
    \item ¿En qué rango de valores se puede encontrar la ABT (biotemperatura media anual)?
    \item Dada una ABT x y una APP z, ¿cuál es el PER (relación de evapotranspiración potencial) correspondiente?
    \item ¿Cuáles son los tipos de regiones latitudinales que existen?
    \item Si una estación climática en la región tropical está a una altitud de 4200 metros, ¿cuál sería su piso altitudinal correspondiente?
\end{enumerate}



















\subsection{Trabajos relacionados}

Resumen de ontologías previas, vocabularios y estándares relevantes (por ejemplo, SSN/SOSA, Darwin Core, GeoSPARQL). Discuta fortalezas y limitaciones, y justifique el reuso y/o alineación.
% Ejemplo de cita: \cite{smith2020ontology}



\subsection{Casos de uso}
Describa escenarios de uso, fuentes de datos y actores.

\section{Metodología}
Describa el proceso (p.~ej., METHONTOLOGY/NeOn): especificación de requisitos, conceptualización, formalización, implementación, evaluación, mantenimiento y publicación.

\section{Diseño conceptual}
\subsection{Alcance y límites}
Defina contexto, granularidad espacial/temporal y supuestos.

\subsection{Clases principales}
Ejemplos: \term{ZonaBioclimática}, \term{VariableClimática}, \term{Indicador}, \term{Observación}, \term{Estación}, \term{Región}, \term{Taxón}.

\subsection{Propiedades y restricciones}
Cardinalidades, dominios/rangos, axiomas clave (disjunción, equivalencias).

\section{Reuso y alineación}
Alinee con vocabularios externos (p.~ej., \textit{SOSA/SSN}, \textit{GeoSPARQL}). Documente mapeos y decisiones.

\section{Implementación en OWL}
\subsection{Prefijos y espacios de nombres}
\noindent
\begin{lstlisting}[language=Turtle,caption={Prefijos base en Turtle.}]
@prefix :      <https://ejemplo.org/bioclima#> .
@prefix owl:   <http://www.w3.org/2002/07/owl#> .
@prefix rdf:   <http://www.w3.org/1999/02/22-rdf-syntax-ns#> .
@prefix rdfs:  <http://www.w3.org/2000/01/rdf-schema#> .
@prefix xsd:   <http://www.w3.org/2001/XMLSchema#> .
@prefix sosa:  <http://www.w3.org/ns/sosa/> .
@prefix geo:   <http://www.opengis.net/ont/geosparql#> .
\end{lstlisting}

\subsection{Clases y propiedades}
\begin{lstlisting}[language=Turtle,caption={Fragmento de clases y propiedades.}]
:ZonaBioclimatica a owl:Class ;
    rdfs:label "Zona bioclimatica"@es .

:VariableClimatica a owl:Class ;
    rdfs:label "Variable climatica"@es .

:Indicador a owl:Class ;
    rdfs:label "Indicador bioclimatico"@es .

:observaVariable a owl:ObjectProperty ;
    rdfs:domain sosa:Observation ;
    rdfs:range  :VariableClimatica ;
    rdfs:label "observa variable"@es .

:valorIndicador a owl:DatatypeProperty ;
    rdfs:domain :Indicador ;
    rdfs:range  xsd:decimal ;
    rdfs:label "valor de indicador"@es .
\end{lstlisting}

\subsection{Individuos de ejemplo}
\begin{lstlisting}[language=Turtle,caption={Individuos ilustrativos.}]
:Aridez a :Indicador ;
    rdfs:label "Indice de Aridez"@es .

:ZB_Mediterranea a :ZonaBioclimatica ;
    rdfs:label "Zona Bioclimatica Mediterranea"@es .
\end{lstlisting}

\section{Consultas SPARQL}
\begin{lstlisting}[language=SPARQL,caption={Zonas bioclimáticas por región.}]
PREFIX :    <https://ejemplo.org/bioclima#>
PREFIX rdfs:<http://www.w3.org/2000/01/rdf-schema#>

SELECT ?zona ?nombre WHERE {
    ?zona a :ZonaBioclimatica ;
                rdfs:label ?nombre .
    # Agregue filtros geoespaciales si aplica (GeoSPARQL)
}
ORDER BY ?nombre
\end{lstlisting}

\begin{lstlisting}[language=SPARQL,caption={Indicadores y valores promedio.}]
PREFIX :   <https://ejemplo.org/bioclima#>
PREFIX xsd:<http://www.w3.org/2001/XMLSchema#>

SELECT ?indicador (AVG(?valor) AS ?promedio) WHERE {
    ?i a :Indicador ;
         rdfs:label ?indicador ;
         :valorIndicador ?valor .
}
GROUP BY ?indicador
\end{lstlisting}

\section{Evaluación}
Verificación (consistencia, coherencia, competencia) y validación (con expertos). Use reasoners (HermiT, Pellet) y pruebas contra CQs.

\section{Publicación y acceso}
Estrategia FAIR: documentación, serializaciones (TTL, RDF/XML), resolución de URIs, endpoint SPARQL, metadatos VoID/DCAT, versionado.

\section{Resultados y discusión}
Análisis de cobertura, utilidad en casos de uso, limitaciones y amenazas a la validez.

\section{Conclusiones y trabajo futuro}
Síntesis de aportes, líneas de mejora, mantenimiento y extensiones planificadas.

\section*{Agradecimientos}
Reconocimientos a colaboradores, instituciones y fuentes de datos.

\appendix
\section{Anexo A: Figuras y tablas}
% Ejemplo de figura (descomente e incluya un recurso válido)
% \begin{figure}[h]
%   \centering
%   \includegraphics[width=0.8\linewidth]{figuras/mapa_zonas.png}
%   \caption{Mapa de zonas bioclimáticas.}
%   \label{fig:mapa-zonas}
% \end{figure}

\appendix
\section{Anexo B: Variables climáticas}
Tabla con variables climáticas modeladas en la ontología. Se indica nombre, abreviatura y unidad.
\begin{table}[h]
    \centering
    \caption{Variables climáticas modeladas.}
    \begin{tabular}{@{}lll@{}}
        \toprule
        Variable & Abreviatura & Unidad \\
        \midrule
        Latitud & & \si{\degree} \\
        Longitud & & \si{\degree} \\
        Altitud (sobre el nivel del mar) & & \si{\meter} \\
        Biotemperatura media anual & ABT & \si{\degreeCelsius} \\
        Precipitación media anual & APP & \si{\milli\metre} \\
        Relación de evapotranspiración potencial & PER & \textit{adimensional} \\
        Evapotranspiración potencial & APE & \si{\milli\metre} \\
        \bottomrule
    \end{tabular}
\end{table}

\clearpage
\printbibliography

\end{document}